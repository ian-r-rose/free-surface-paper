\documentclass[a4paper,12pt]{article}
\usepackage[utf8]{inputenc}

%opening
\title{Response to referees}
\author{Ian Rose, Bruce Buffett, and Timo Heister}
\date{}

\begin{document}

\maketitle

We thank referees Fabio Crameri and Dave May for their thoughtful and constructive reviews,
which are appended to this document for convenience.

\subsection*{Response to reviewer 1}

Fabio Crameri provided a number of helpful comments and suggestions for clarification,
which we summarize here:

\begin{itemize}
  \item We have identified $\mathbf{I}$ as the identity tensor.
  \item We have itentified $\mathbf{n}$ as the unit normal.
  \item We have clarified lines 154-155, 174 and 460-463.
  \item We have improved the caption for Figure 5.
\end{itemize}


\subsection*{Response to reviewer 2}

Dave May gave a very thorough review with substantial suggestions for 
improvements and additions to the manuscript, which we address in detail here.

\subsubsection*{Summary}

May's review has two main criticisms:
\begin{enumerate}
\item That we do not provide sufficient non-trivial examples of the usage of our new 
  time integratior, especially making detailed comparisons with the existing quasi implicit (QI)
  time integrator of Kaus et. al. (2010), including a discussion of the relative advantages
  and disadvantages of the scheme.
\item That we have not provided enough information of how to determine the stabilization timescale ($\tau^*$)
  which is required to use the nonstandard finite difference (NSFD) time integrator, including
  a discussion of the consequences of making a poor choice for $\tau^*$.
\end{enumerate}

In order to address these we have made two major additions to the manuscript:

First, we have added a more lengthy discussion of how to numerically determine the minimum relaxation time of
a given geodynamic simulation, which is the best choice for $\tau^*$ (see Section 6.4). Our favored method
is using a power iteration, which finds the dominant eigenvector/eigenvalue pair of the system.
Each iteration involves a solution of a Stokes system, and we find good a approximation for the minimum
relaxation time in 5-20 iterations, depending on how good the initial guess is.

This estimate of the minimum relaxation time is not only useful for the NSFD scheme,
as it can also be used to determine appropriate timesteps for quasi-implicit time integration,
or even forward Euler time integration. 
We provide several numerical examples of using power iteration, which confirm our analysis
of the effect of the quasi-implicit stabilization term on the spectrum of the system (Equation 29).
The results of this test are shown in Figure 2.

Our second major addition is an investigation of the Rayleigh-Taylor instability with a free surface (Section 8.2),
which was considered in Kaus et. al. (2010). We found that the QI and NSFD schemes are comparably
accurate the same (adaptively chosen) timestep (with a slight advantage to QI). However, we have found
that the NSFD scheme is able to take larger stable timesteps (though at the cost of lower accuracy).

We have also used the Rayleigh-Taylor benchmark to investigate the effect of different choices of 
the stabilization parameter $\tau^*$. We considered four cases: 
\begin{enumerate}
  \item recomputing the minimum relaxation timescale every 50 timesteps and setting $\tau^*$ to that value 
    (this allows for changes in the viscosity and density structure with time). This is used as the reference case.
  \item Using the initial value of the minimum relaxation time for the whole simulation.
  \item Using a value that is about a factor of two too large.
  \item Using a value that is about a facator of two too small.
\end{enumerate}
The first two cases are almost identical, while the third and fourth have significant differences with the reference case. 

We also have specific responses to the ``General Comments'' and ``Detailed Corrections'' sections of
May's review, detailed below.

\subsubsection*{General comments}
\begin{enumerate}
  \item The manuscript is now numbered.
  \item We have made equation referencing consistent.
  \item Equations are now punctuated.
  \item Section headings now all use sentence case.
  \item ``timestep'' is now used uniformly.
  \item The point about the title is well-taken, though we would argue that our analysis of the stability of the
system and the effect of QI stabilization on the spectrum are at least as important as our new time integrator.
We have retitled the manuscript ``Stability and accuracy of free surface time integration in viscous flows.''
  \item We have added a detailed look at the time dependence of $\tau^*$. In many cases its time variation is
probably not large enough to merit recomputing it during the course of a simulation.
  \item We have provided new numerical experiments investigating the numerical determination of $\tau^*$, and have
taken a detailed look at a Rayleigh-Taylor benchmark. In the Rayleigh-Taylor case we have QI and NSFD schemes,
investigated the time dependence of $\tau_\mathrm{min}$, and demonstrated the effects of a poor choice for $\tau_\mathrm{min}$.
We ran experiments for the Schmeling et. al. (2008) benchmark, but have ultimately not included them because 
(1) they did not provide new insight over the Rayleigh-Taylor benchmark, and 
(2) this particular benchmark has not proved to be particularly useful due to its extremely sensitive dependence 
on model parameters and viscosity averaging schemes, making the results difficult to interpret.
  \item We have added a treatment of using power iteration to calculate the minimum relaxation time, which is
useful not just for determining $\tau^*$, but can also be of use for analysis of timestepping with other schemes.
  \item We have added some discussion of direct comparision of the QI and NSFD schemes. We agree that the benefits
    of spatial adapdivity are less connected with the preceding discussion on timestepping, however we still felt
    that demonstrating the savings of combining AMR with a free surface had some value.
\end{enumerate}

\subsubsection*{Detailed corrections}
\begin{enumerate}
  \item We have reworded the abstract to emphasize time integration of the free surface.
  \item We have cited Jarvis and Peltier (1982) for an early dynamic topography calculation.
  \item We have given an additional reference for the sticky air method.
  \item The surface of Earth is not truly stress free, since there are stresses due to oceans and atmospheres.
    We would argue that ``more closely matches'' is appropriate.
  \item We have cited Popov and Sobolev (2008) for another implicit approach to free surfaces.
  \item We have explicitly given the conditions for free-slip, free surface, and Dirichlet boundary conditions.
  \item We agree that discussion of the discrete mesh is more appropriate in Section 7, and have reorganized accordingly.
  \item This is a good suggestion. We have reorganized Section 3 to put all the information dealing with
    the time evolution of $\Omega(t)$ in one place.
  \item We have introduced a new notation $\hat{\Omega}$ for when we are dealing with a spatially discretized domain.
However, for most of the analysis of Sections 2-5 the equations are agnostic to whether we are using a discretized
or continuous domain, so we have retained the continuous notation when appropriate. We have also tried to clarify
what is meant by the prediction of the shape at a later time $\Delta \Omega$ as an approximate integration of the
domain evolution equation (Eq. 7).
  \item We have changed line 178 to ``become unstable''.
  \item We have clarified the approximations under which the QI stabilization term is identical to
the $M$ matrix in the homogeneous problem. Specifically, they are approximately the same when the slope is small,
and density variations are small compared to the reference value.
  \item We have added a note to this effect.
  \item It is true that our ALE formulation involves assembly of a surface mass matrix, as does the solution of
    the generalized eigenvalue problem. However, not every implementation need have the same approach. That being
    said, it is a good point that the modifications to the Stokes system matrix are overall not very difficult to apply.
  \item We have rephrased this point.
  \item $x$ and $y$ axes are now italicized.
  \item We have clarified this point.
  \item This has been fixed.
  \item We mean Equation (16), which can be reorganized to Equation (18), so either are appropriate here.
  \item We have clarified this point in the text, as well as explored by how much $\tau_\mathrm{min}$ might change.
  \item Equation (19) involves the inverse of $\tau_i$, so finding the largest eigenvalue of this system
is equivalent to finding the minimum relaxation time.
  \item Boundary conditions are now all defined mathematically.
  \item Absolutely right, this is now rephrased.
  \item This is now clarified to an ``$L_2$ projection of $u \cdot n$ onto $\Gamma_F$''.
  \item Changed to say ``cheap''.
  \item Line 267 now says ``acceleration due to gravity''.
  \item L2  $\rightarrow L_2$.
  \item We now use the symbol $E$ for the error. 
  \item Line 472-473 has been rephrased.
  \item We now give more information about the parameters used to run this simulation, including refinement schemes, 
    maximum spatial resolution, and timestepping schemes.
  \item It is true that 500 years is a very small timestep, and that forward Euler timestepping would be stable for this.
We use this timestep because Crameri et. al. (2012) found that a timestep of several hundred years was needed
for the solution to fully converge. The purpose of this test is less to investigate the time integration schemes
than to demonstrate the effectiveness of combining adaptive mesh refinement with a free surface treatment.
To a certain extent we agree that this is not the primary point of the manuscript. However, we still think it
is valuable to demonstrate that our implementation, which is open source and freely available, has this capability.
  \item QI is now defined.
  \item The caption now reads ``significantly fewer unknowns''.
\end{enumerate}


\end{document}
