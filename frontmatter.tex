\chapter{Time integration of free surfaces in viscous flows}
\label{chap:free_surface}

Geodynamic simulations increasingly rely on simulations with a true free surface to 
investigate questions of dynamic topography, tectonic deformation, gravity perturbations, and
global mantle convection. However, implementations of free surface boundary conditions 
have proven challenging from a standpoint of accuracy, robustness, and stability.
In particular, time integration of a free surface tends to suffer from a numerical instability
that manifests as sloshing surface motions, also known as the ``drunken sailor'' instability.
This instability severely limits stable timestep sizes to those much smaller than could be used
in geodynamic simulations without a free surface. 
Several schemes have been proposed in the literature to deal with these instabilities.

Here we analyze the problem of creeping viscous flow with a free surface and discuss the 
origin of these instabilities. We demonstrate their cause and how existing stabilization 
schemes work to damp them out.
We also propose a new scheme for removing instabilities from free surface calculations. 
It does not require modifications to the system matrix, nor additional variables, but is instead
an explicit scheme based on nonstandard finite differences.  It relies on a single 
stabilization parameter which may be identified with the smallest relaxation timescale of the
free surface.

Finally, we discuss the implementation of a free surface in the open source, community based
mantle convection software \texttt{ASPECT}.
