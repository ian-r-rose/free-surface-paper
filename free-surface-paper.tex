%% 
%% Copyright 2007, 2008, 2009 Elsevier Ltd
%% 
%% This file is part of the 'Elsarticle Bundle'.
%% ---------------------------------------------
%% 
%% It may be distributed under the conditions of the LaTeX Project Public
%% License, either version 1.2 of this license or (at your option) any
%% later version.  The latest version of this license is in
%%    http://www.latex-project.org/lppl.txt
%% and version 1.2 or later is part of all distributions of LaTeX
%% version 1999/12/01 or later.
%% 
%% The list of all files belonging to the 'Elsarticle Bundle' is
%% given in the file `manifest.txt'.
%% 
%% Template article for Elsevier's document class `elsarticle'
%% with harvard style bibliographic references
%% SP 2008/03/01

\documentclass[preprint,12pt,authoryear]{elsarticle}

%% Use the option review to obtain double line spacing
%% \documentclass[authoryear,preprint,review,12pt]{elsarticle}

%% Use the options 1p,twocolumn; 3p; 3p,twocolumn; 5p; or 5p,twocolumn
%% for a journal layout:
%% \documentclass[final,1p,times,authoryear]{elsarticle}
%% \documentclass[final,1p,times,twocolumn,authoryear]{elsarticle}
%% \documentclass[final,3p,times,authoryear]{elsarticle}
%% \documentclass[final,3p,times,twocolumn,authoryear]{elsarticle}
%% \documentclass[final,5p,times,authoryear]{elsarticle}
%% \documentclass[final,5p,times,twocolumn,authoryear]{elsarticle}

%% For including figures, graphicx.sty has been loaded in
%% elsarticle.cls. If you prefer to use the old commands
%% please give \usepackage{epsfig}

%% The amssymb package provides various useful mathematical symbols
\usepackage{amssymb}
\usepackage{amsmath}

\newif\ifdetail
\detailfalse
%\detailtrue

%% The amsthm package provides extended theorem environments
%% \usepackage{amsthm}

%% The lineno packages adds line numbers. Start line numbering with
%% \begin{linenumbers}, end it with \end{linenumbers}. Or switch it on
%% for the whole article with \linenumbers.
%% \usepackage{lineno}

\journal{Physics of the Earth and Planetary Interiors}

\begin{document}

\begin{frontmatter}

%% Title, authors and addresses

%% use the tnoteref command within \title for footnotes;
%% use the tnotetext command for theassociated footnote;
%% use the fnref command within \author or \address for footnotes;
%% use the fntext command for theassociated footnote;
%% use the corref command within \author for corresponding author footnotes;
%% use the cortext command for theassociated footnote;
%% use the ead command for the email address,
%% and the form \ead[url] for the home page:
%% \title{Title\tnoteref{label1}}
%% \tnotetext[label1]{}
%% \author{Name\corref{cor1}\fnref{label2}}
%% \ead{email address}
%% \ead[url]{home page}
%% \fntext[label2]{}
%% \cortext[cor1]{}
%% \address{Address\fnref{label3}}
%% \fntext[label3]{}

\title{Free surface computations in mantle convection models}

%% use optional labels to link authors explicitly to addresses:
%% \author[label1,label2]{}
%% \address[label1]{}
%% \address[label2]{}

\author{I. Rose, B. Buffett, and T. Heister}

\address{}

\begin{abstract}
Geodynamic simulations are increasingly turning to simulations with a true free surface to 
investigate questions of dynamic topography, tectonic deformation, gravity perturbations, and
global mantle convection. However, implementations of free surface boundary conditions 
have proven challenging from a standpoint of accuracy, robustness, and stability.

In particular, free surfaces tend to suffer from sloshing instabilities, also known as 
the ``drunken sailor'' instability, which severely limit time step sizes. Several 
schemes have been proposed in the literature to deal with these instabilities.

Here we analyze the problem of creeping viscous flow with a free surface and discuss the 
origin of these instabilities. We demonstrate their cause and how existing stabilization 
schemes work to damp them out.

We also propose a new scheme for removing instabilites from free surface calculations. 
It does not require modifications to the system matrix, nor additional variables, but is instead
an explicit scheme based on nonstandard finite differences.  It relies on a single 
stabilization parameter which may be identified with the smallest relaxation timescale of the
free surface.

Finally, we discuss the implementation of a free surface in the open source, community based
mantle convection software \texttt{Aspect}.
\end{abstract}

\begin{keyword}
%% keywords here, in the form: keyword \sep keyword

%% PACS codes here, in the form: \PACS code \sep code

%% MSC codes here, in the form: \MSC code \sep code
%% or \MSC[2008] code \sep code (2000 is the default)

\end{keyword}

\end{frontmatter}

%% \linenumbers

%% main text
\section{Introduction}
\label{sec:intro}

\section{Stability}
\subsection{Eigenvalue analysis}
\label{sec:eigenvalue}

We begin with the homogeneous Stokes equations for incompressible creeping flow:
\begin{equation}
\nabla \cdot \left( 2 \eta \varepsilon( \mathbf{u} ) \right) - \nabla p = 0
\label{eq:stokes}
\end{equation}

\begin{equation}
\nabla \cdot \mathbf{u} = 0
\label{eq:incompressible}
\end{equation}
where $\mathbf{u}$ is the velocity, $p$ is the pressure, $\eta$ is the viscosity, and 
$\varepsilon(\mathbf{u}) = \frac{1}{2}(\nabla \mathbf{u} + (\nabla \mathbf{u} )^T )$ is the strain-rate tensor.

We will proceed with this analysis within a finite element framework, though similar arguments should 
work for other discrete methods.
We transform the governing equations into the weak form amenable to finite elements by standard procedures \citep[e.g.][]{zienkiewicz1977finite} to get

\begin{equation}
-\int_\Omega 2 \eta \varepsilon( \mathbf{w} ) \colon \varepsilon( \mathbf{u} ) + \int_\Omega p \nabla \cdot \mathbf{w} 
+ \int_{\partial \Omega_N} \mathbf{w} \cdot \mathbf{T} \cdot \mathbf{n} = 0 
\label{eq:weak_stokes}
\end{equation}

\begin{equation}
\int_\Omega q \nabla \cdot \mathbf{u} = 0
\label{eq:weak_incompressible}
\end{equation}
where $\mathbf{w}$ and $q$ are suitably chosen test functions and the integrals over 
$\Omega$ and $\partial \Omega_N$ are over the volume of the domain and the part of the boundary with stress boundary conditions, respectively.
$\mathbf{T}$ is the stress tensor, defined by
\begin{equation}
\mathbf{T} = 2 \eta \varepsilon(\mathbf{u}) - p \mathbf{I}
\end{equation}
The integral over the surface in Equation~\eqref{eq:weak_stokes} accounts for boundary stresses, 
which should be zero when evaluated on a true free surface.

Rather than analyzing finite deformation to the free surface, we will make the analytically useful
approximation of small deformations about the hydrostatic reference surface and analyze their stability.
We will therefore  evaluate the integrals in Equation~\eqref{eq:weak_stokes} 
over the hydrostatic reference surface and introduce an auxilliary variable $\zeta$ which 
represents the (small) topography of the free surface relative to that reference surface.
The equation for time evolution of the free surface is then
\begin{equation}
\frac{d \zeta}{dt} = \mathbf{u \cdot n}
\label{eq:surface_evolution}
\end{equation}
where $\mathbf{u}$ is evaluated at the free surface.
The stress on this reference surface can be approximated by the hydrostatic pressure
\begin{equation}
\mathbf{T} \approx \rho g \zeta \mathbf{I}
\label{eq:hydrostatic}
\end{equation}
where $\rho$ is the density and $g$ is gravity. Equation~\eqref{eq:weak_stokes} then becomes

\begin{equation}
-\int_\Omega 2 \eta \varepsilon( \mathbf{w} ) \colon \varepsilon( \mathbf{u} ) + \int_\Omega p \nabla \cdot \mathbf{w} 
+ \int_{\partial \Omega_N} \rho g \zeta  \mathbf{w} \cdot \mathbf{n} = 0 
\end{equation}

We would like to analyze the time evolution of the normal modes of this system: each mode 
is the relaxation of topography with a characteristic relaxation time.  
We denote the normal modes by $\left[ \mathbf{u}_i, p_i, \zeta_i \right]^T$, with
relaxation times $\tau_i$, where the subscript corresponds to the $i$th normal mode.

The equations decouple for the normal modes, and Equation~\eqref{eq:surface_evolution} then becomes

\begin{equation}
\frac {d}{d t} \zeta_i(\mathbf{x},t) = \frac{d}{dt} \zeta_i(\mathbf{x})e^{-t/\tau_i} = -\frac{\zeta_i(\mathbf{x},t)}{\tau_i} = \mathbf{u}_i \cdot \mathbf{n}
\end{equation}
This can then be used to eliminate $\zeta$ from the Stokes system:
\begin{equation}
-\int_\Omega 2 \eta \varepsilon( \mathbf{w} ) \colon \varepsilon( \mathbf{u}_i ) + \int_\Omega p_i \nabla \cdot \mathbf{w} 
= \tau_i \int_{\partial \Omega_N} \rho g (\mathbf{u}_i \cdot \mathbf{n} ) (\mathbf{w} \cdot \mathbf{n})
\label{eq:weak_eigen}
\end{equation}
When these equations are discretized \citep[e.g.][]{kronbichler2012high} we get
\begin{equation}
\begin{bmatrix}
A & B^T \\
B & 0 \\
\end{bmatrix}
\begin{bmatrix}
\mathbf{u}^n_i \\
p^n_i
\end{bmatrix}
=
\tau_i
\begin{bmatrix}
M & 0 \\
0 & 0
\end{bmatrix}
\begin{bmatrix}
\mathbf{u}^n_i \\
p^n_i
\end{bmatrix}
\label{eq:generalized_eigenvalue}
\end{equation}
where $\mathbf{u}^n_i$ and $p^n_i$ are finite-dimensional representations of $\mathbf{u}_i$ and $p_i$,
and $M$ is the discretization of the bilinear form on the right-hand side of Equation~\eqref{eq:weak_eigen}.

Equation~\eqref{eq:generalized_eigenvalue} is a generalized eigenvalue problem for the normal modes of the system.
It is rather more difficult to solve than a standard eigenvalue problem because the matrix on the right-hand-side 
is not invertible. It may, however, be transformed into a standard eigenvalue problem.
Define
\begin{equation}
\begin{aligned}
C &= 
\begin{bmatrix}
A & B^T \\
B & 0 \\
\end{bmatrix} \\
D &= 
\begin{bmatrix}
M & 0 \\
0 & 0
\end{bmatrix} \\
\mathbf{y}_i &= 
\begin{bmatrix}
\mathbf{u}^n_i \\
p^n_i
\end{bmatrix} \\
\end{aligned}
\end{equation}
Then multiplying both sides by $C^{-1}$ we get
\begin{equation}
\tau_i (C^{-1}D)\mathbf{y}_i = \mathbf{y}_i
\label{eq:standard_eigenvalue}
\end{equation}

This eigenvalue equation can be solved for the normal modes and relaxation times of the Stokes system with 
a free surface.

\subsection{Time integration of the free surface}
\label{sec:timestepping}

Armed with the normal modes and relaxation times of the Stokes system, we can write down the
formal solution to Equation~\ref{eq:surface_evolution}. Let the initial surface topography be 
represented by a linear combination of its normal modes
\begin{equation}
\zeta(\mathbf{x}, t=0) = \displaystyle \sum_i a_i \zeta_i(\mathbf{x})
\end{equation}
then the time evolution of the free surface is given by
\begin{equation}
\zeta( \mathbf{x}, t) = \displaystyle \sum_i a_i \zeta_i(\mathbf{x}) e^{-t/\tau_i}
\end{equation}

Of course, most finite element (or finite difference, or finite volume) geodynamic simulations do not resolve 
the solution and surface topography into its normal modes and integrate those separately. 
Indeed, analytical solutions for the normal modes are only known for simple geometries and rheologies.
Instead, they attain a velocity solution and simply advect the free surface with the local velocity.
The normal mode solution is instructive, however. 
Each mode has the form of a decay equation with characteristic decay time $\tau_i$.
The decay equation is the archetypical example of a stiff ordinary differential equation.
If we were to numerically integrate this in time with a forward Euler method, we would find the 
time-step criterion for stability \citep[e.g.][]{leveque2007finite} to be
\begin{equation}
\Delta t  \le 2 \tau_{\mathrm{min}}
\label{eq:cfl_euler}
\end{equation}
The maximum allowable timestep is limited by the minimum relaxation timescale.
If a larger timestep than this is taken then those modes will go unstable.
The modes with the smallest relaxation times are usually those with the largest lengthscales \citep{schubert2001mantle}, 
and so it will be those which go unstable first, a phenomenon which has been called 
a ``sloshing'', or ``drunken sailor'' instability \citep{kaus2010stabilization}.

\subsection{Quasi-implicit stabilization}
\label{sec:kmm}

The relaxation timescales for surface topography tend to be considerably longer than those for 
convection or tectonic deformation, so the stability requirements for a forward Euler scheme
are quite onerous.  On the other hand, an implicit time marching scheme requires solving 
a nonlinear system for the new surface position, or assembling a larger system with surface
topography unknowns \citep[e.g.][]{kramer2012implicit}.

\citet{kaus2010stabilization} proposed a scheme whereby the body forces on the domain are 
evaluated on a prediction of the shape of the domain at a later time.
The weak form of the right-hand-side body forces in the finite element discretization is then

\begin{equation}
\mathbf{f}_{\mathrm{body}} = \int_{\Omega + \Delta \Omega} \rho \mathbf{w} \cdot \mathbf{g}
\label{eq:predict}
\end{equation}

The shape prediction can be approximated by integration of the velocity:
\begin{equation}
\Delta \Omega \approx \theta \Delta t \mathbf{u}
\end{equation}
where $\theta$ is a free parameter that corresponds to the magnitude of the 
correction, where zero is no stabilization and one is fully (quasi) implicit.

One can approximately expand the integral in Equation~\eqref{eq:predict} using 
Reynold's transport theorem to find

\begin{equation}
\int_{\Omega + \Delta \Omega} \rho  \mathbf{w} \cdot \mathbf{g} \approx
\int_{\Omega} \rho  \mathbf{w} \cdot \mathbf{g} + \theta \Delta t \int_{\partial \Omega} \rho ( \mathbf{w \cdot g}) (\mathbf{u \cdot n} )
\label{eq:kmm}
\end{equation}
The volume integral is the same as that of the unstabilized problem, but now there is a surface integral.
The surface integral has the form of a velocity-dependent surface stress pushing down on the 
domain, and can be thought of as an artificial viscous damping of the surface.
Since it depends on the velocity, it can no longer go on the right-hand side of the 
equation, and must be added as a stabilization term in the system matrix.  
Empirically it has been found to be successful at damping instabilities \citep{kaus2010stabilization, quinquis2011role, crameri2012comparison}.

Indeed, using the formalism in Section~\ref{sec:eigenvalue} we can see the effect of this 
stabilization term. In the reference state we note that $\mathbf{g} = g \mathbf{n}$, which
allows us to write the stabilization term as
\begin{equation}
\theta \Delta t \int_{\partial \Omega} \rho g ( \mathbf{w \cdot n}) (\mathbf{u \cdot n} )
\end{equation}

The integral here is precisely the same as that on the right hand side of Equation~\eqref{eq:weak_eigen}, which was discretized as the matrix $M$.

Indeed, if we discretize the Stokes system with the quasi-implicit stabilization term, we find
a new generalized eigenvalue problem:

\begin{equation}
\begin{bmatrix}
A + \theta \Delta t M & B^T \\
B & 0 \\
\end{bmatrix}
\begin{bmatrix}
\mathbf{u}^n_i \\
p^n_i
\end{bmatrix}
=
\tau^S_i
\begin{bmatrix}
M & 0 \\
0 & 0
\end{bmatrix}
\begin{bmatrix}
\mathbf{u}^n_i \\
p^n_i
\end{bmatrix}
\label{eq:generalized_eigenvalue}
\end{equation}
where $\tau^S_i$ indicate the eigenvalues of the stabilized system.
This system may be rearranged:
\begin{equation}
\begin{bmatrix}
A & B^T \\
B & 0 \\
\end{bmatrix}
\begin{bmatrix}
\mathbf{u}^n_i \\
p^n_i
\end{bmatrix}
=
\left(\tau^S_i - \theta \Delta t \right)
\begin{bmatrix}
M & 0 \\
0 & 0
\end{bmatrix}
\begin{bmatrix}
\mathbf{u}^n_i \\
p^n_i
\end{bmatrix}
\label{eq:generalized_eigenvalue}
\end{equation}

This is precisely the same generalized eigenvalue problem as Equation~\eqref{eq:generalized_eigenvalue},
so its eigenvalues must be the same.  This allows us to write the eigenvalues of the stabilized problem in
terms of those of the unstabilized problem:
\begin{equation}
\tau^S_i = \tau_i + \theta \Delta t
\end{equation}

Essentially, the stabilization term lengthens every relaxation time for the system by an amount $\theta \Delta t$.
This correspondingly lengthens the maximum stable timestep for the forward Euler method:
\begin{equation}
\Delta t  \le \frac{2}{1-\theta} \tau_{\mathrm{min}}
\label{eq:cfl_euler_stabilized}
\end{equation}
note that as $\theta$ goes to one this scheme should become unconditionally stable,
but nonlinear effects and discretization errors due to finite deformation of the surface could 
prevent that stability.

\subsection{A novel time-integration scheme}

\subsubsection{Nonstandard finite-differences}
One downside of the quasi-implicit scheme is that it requires a modification of the system matrix, 
and as we have seen this shifts its spectrum.
An alternative would be to construct a time-integration scheme with better stability properties.  
The reason that forward Euler integration performs so badly with the decay of topography is that 
at larger timesteps it overshoots its equilibrium position. This overshoot causes it to lurch back 
to the \emph{other} side of equilibrium, overshooting even more.
We would like to construct an explicit scheme that accounts for following properties which we know the system has:
\begin{itemize}
\item In the absence of forcing, topography always decreases.
\item Relaxation of small amplitude topography takes the form of exponential decay.
\item The decay mode with the shortest relaxation time is the least stable.
\end{itemize}

A general expression for the evolution of a free surface from time $t=0$ to time $t=\Delta t$ is 
\begin{equation}
\mathbf{x}(\Delta t) = \mathbf{x}(0) + \int_0^{\Delta t} \mathbf{u}(t) dt
\label{eq:time_integration}
\end{equation}

where $\mathbf{x}$ is the location of the free surface.  If we approximate $\mathbf{u}(t) \approx \mathbf{u}(0)$, 
we of course recover the forward Euler scheme.
However, we can make another choice based on our knowledge of the system behavior. 
We can approximate $\mathbf{u}(t)$ as
\begin{equation}
\mathbf{u}(t) = \mathbf{u}(0) e^{-t/\tau^*}
\label{eq:velocity_decay}
\end{equation}
where $\tau^*$ is some as-yet undetermined positive constant.
This form of $\mathbf{u}$ automatically decays in time, and as we shall see, has much better 
stability properties than forward Euler integration.
Plugging Equation~\eqref{eq:velocity_decay} into Equation~\eqref{eq:time_integration} and integrating, we find

\begin{equation}
\mathbf{x}(\Delta t) = \mathbf{x}(0) + \mathbf{u}(0) \tau^* \left(1-e^{-\Delta t/\tau^*} \right)
\label{eq:nsfd}
\end{equation}

The quantity $\tau^*(1-e^{-\Delta t / \tau^*})$ acts as a pseudo-timestep for advecting the free-surface.
Equation~\eqref{eq:nsfd} is what is known as a nonstandard finite-difference model, based on
constructing unusual discrete models for differential equation integration.
The theory has been developed in, among others, a series of papers by
\citet{mickens1994nonstandard, mickens2002nonstandard, mickens2005dynamic}.

\subsubsection{Stability of the scheme}
The parameter $\tau^*$ sets how quickly $\mathbf{u}$ decays in Equation~\eqref{eq:velocity_decay}, and a good 
choice is crucial for accuracy and stability. A shorter decay time corresponds to more stabilization,
but if it becomes too short, the surface velocity will become overdamped. In general, we will want 
to choose $\tau^*$ so that it is as close as possible to the least stable mode, or $\tau_{\mathrm{max}}$.

To investigate the stability of this scheme we consider a velocity solution with power in 
a single normal mode of the system $\mathbf{u} = a_i \mathbf{u}_i$.
This will decay exponentially with relaxation time $\tau_i$, or
\begin{equation}
\frac{d a_i} {dt} = - \frac{ a_i }{\tau_i} 
\end{equation}
Applying the nonstandard finite difference scheme, we find
\begin{equation}
a_i^{(1)} = a_i^{(0)} \left[ 1 - \frac{\tau^*}{\tau_i} \left(1-e^{-\Delta t/\tau^*} \right) \right]
\label{eq:recursion}
\end{equation}
In order for the scheme to be stable, the quantity in brackets must not blow up as it is repeatedly 
multiplied by itself, or 
\begin{equation}
\left| 1 - \frac{\tau^*}{\tau_i} \left(1-e^{-\Delta t/\tau^*} \right) \right| \le 1
\end{equation}

\ifdetail
Taking the positive value of the absolute value yields the criterion of 
\begin{equation}
- \frac{\tau^*}{\tau_i} \left(1-e^{-\Delta t/\tau^*} \right) \le 0
\end{equation}

which can be rearranged to find
\begin{equation}
e^{-\Delta t/\tau^*} \le 1
\end{equation}

which is simply a statement that the timestep must be postive.

Taking the negative value of the absolute value yields
\begin{equation}
  \frac{\tau^*}{\tau_i} \left(1-e^{-\Delta t/\tau^*} \right)  \le 2
\end{equation}

\begin{equation}
 1 - 2 \frac{\tau_i}{\tau^*} \le e^{-\Delta t/\tau^*}
\end{equation}

if the left hand side is less than zero, then this is true regardless of step size, or
\begin{equation}
\tau^* \le 2 \tau_i
\end{equation}

if the left hand side is between zero and one, the expression is more complicated. 
Taking the log of both sides:

\begin{equation}
\Delta t \le -\tau^* \log \left(1 - 2 \frac{\tau_i}{\tau^*} \right)
\end{equation}

It is convenient to write this in terms of dimensionless times $\Delta t/ \tau_i$ and $\tau^*/\tau_i$.

\begin{equation}
\frac{\Delta t}{\tau_i} \le -\frac{\tau^*}{\tau_i} \log \left(1 - 2 \frac{\tau_i}{\tau^*} \right)
\end{equation}

\fi

The stability of this scheme is determined by the choice of $\tau^*$.  
It has a region of unconditional stability, where
\begin{equation}
\tau^* \le 2 \tau_i
\end{equation}
Otherwise, the scheme is conditionally stable, with the timestep limited by 
\begin{equation}
\Delta t \le -\tau^* \log \left(1 - 2 \frac{\tau_i}{\tau^*} \right)
\end{equation}
The stability region is plotted in Figure \ref{fig:stability}. 

\begin{figure}
\includegraphics[width=0.9\textwidth]{figures/stability.pdf}
\label{fig:stability}
\caption{Left: stability region for the nonstandard finite difference scheme (green). On the x-axis is the value of the stabilization time, in units of the minimum relaxation time $\tau_{\mathrm{min}}$.  On the y-axis is the value of the timestep, also in units of the minimum relaxation time. For $\tau^*\le \tau_{\mathrm{min}}$ it is unconditionally stable.  Right: stability region for the quasi-implicit scheme (green).  Again, the y axis is in units of the minumum relaxation time.  The x-axis shows the value of the stabilization parameter $\theta$. For $\theta = 1$ it is unconditionally stable. In both cases the stability region for the forward Euler scheme is also shown in blue.}
\end{figure}

\subsubsection{Accuracy of the scheme}
It is important to note that even though we derived the nonstandard finite difference scheme assuming a decaying exponential
for $\mathbf{u}$, it is formally a first-order accurate scheme. As such, it is capable of capturing arbitrary motions 
of the free surface, but with good stability properties.
Again we may take one of the normal modes as an example, differencing the analytical solution from the nonstandard finite-difference scheme:

\begin{equation}
\begin{aligned}
a_i(\Delta t) - a_i^{(1)} &= a_i(0) e^{-\Delta t/\tau_i} - a_i{(0)} \left[ 1 - \frac{\tau^*}{\tau_i} \left(1-e^{-\Delta t/\tau^*} \right) \right] \\
                          &= a_i{(0)} \left[ e^{-\Delta t/\tau_i} - 1 + \frac{\tau^*}{\tau_i} \left(1-e^{-\Delta t/\tau^*} \right) \right] \\
\end{aligned}
\end{equation}

Note that when $\tau_i = \tau^*$ the nonstandard finite difference scheme is exact.
The exponentials may be expanded to find
\begin{equation}
a_i(\Delta t) - a_i^{(1)} = a_i(0) e^{-\Delta t/\tau_i} - a_i{(0)} \left[ \left(\frac{\Delta t}{\tau_i}\right)^2 - \frac{\tau^*}{\tau_i} \left(\frac{\Delta t }{\tau^* }\right)^2 \right] 
\end{equation}

\subsubsection{Choice of $\tau^*$}
As discussed above, a full geodynamic simulation will have a spectrum of relaxation times.
For complete stability, the parameter $\tau^*$ should be chosen such that every mode is stable.
In practice, this means that a good choice is 
\begin{equation}
\tau^* \approx \tau_{\mathrm{min}}
\label{eq:tau_choice}
\end{equation}

Unfortunately, for many models the minimum relaxation time will not be known beforehand. 
In order to use the nonstandard finite difference scheme, the value of $\tau^*$ would need 
to be calculated or estimated first.  There are several possible ways to determine this value:

\begin{itemize}
\item Direct solution of Equation~\eqref{eq:generalized_eigenvalue}. This can be expensive, although
only the minimun relaxation time is required. For certain geometries a simple power iteration on 
the standard eigenvalue problem~\eqref{eq:standard_eigenvalue} could be enough.
\item Analytical formulae. Several geometry and viscosity model combinations have analytical solutions
for relaxation spectra. Even if the user's model is not precisely the same (e.g., has some lateral viscosty
variations), an analytical approximation may be sufficient.
\item Scaling. In general, we expect the relaxation times to scale with $\tau \sim \frac{\eta}{\rho g L}$,
where the density, gravity, viscosity, and lengthscale are all representative values.  A small amount
of experimentation near this value of $\tau$ can find an appropriate relaxation time.
\item Observation of instabilities.  The mere act of observing a sloshing instability can furnish 
an estimate of its relaxation time.
\end{itemize} 


\section{Implementation in \texttt{Aspect} }
We have implemented the ability to run free surface simulations in the mantle convection software \texttt{Aspect} \citep{kronbichler2012high}. 
\texttt{Aspect} is designed to be highly flexible and modular, with the ability for user-defined rheologies, geometries, and gravity models. 
The implementation of the free surface, therefore, needs to be general enough to work for many combinations of these models, including those 
which may not have been written yet. In particular, it cannot rely on assumptions regarding the shape of the domain.

Furthermore, \texttt{Aspect} allows for computations in both 2D and 3D, so the implementation must be sufficiently dimension
independent to work in both cases. We implement the free surface within an arbitrary Lagrangian-Eulerian (ALE) formulation
\citep[e.g][]{fullsack1995arbitrary,donea2004encyclopedia}.

\texttt{Aspect} is a parallelized, distributed memory code with adaptive mesh refinement.
The free surface implementation works with these features.  We store the mesh vertex positions 
in a fully distributed vector. This vector is continually updated and redistributed across 
processes upon mesh adaptation (which is handled by the adaptive octree library \texttt{p4est}).
We also provide adaptive refinement indicators based on being near to the free surface or 
when the free surface slope is steep to allow for accurate interface tracking.


\subsection{Remeshing}
In ALE calculations the internal mesh velicity is undetermined.
In general, one wants to smoothly deform the mesh so as to preserve its regularity, 
avoiding inverted or otherwise poorly conditioned cells.
The mesh deformation can be calcluated in many different ways, including algebraic \citep[e.g.][]{thieulot2011fantom} 
and PDE based approaches.

We choose to implement remeshing based on solving Laplace's equation for the mesh velocity.
We solve the equation
\begin{equation}
\nabla^2 \mathbf{u}_{\mathrm{mesh}} = 0
\label{eq:laplacian_smoothing}
\end{equation}
subject to the boundary conditions
\begin{equation}
\begin{aligned}
&\mathbf{u}_\mathrm{mesh} = \left( \textbf{u} \cdot \textbf{n} \right) \textbf{n} & \qquad & \textrm{on } \partial \Omega_{\textrm{free surface}}, \\
&\mathbf{u}_\mathrm{mesh} \cdot \textbf{n} = 0 & \qquad & \textrm{on } \partial \Omega_{\textrm{free slip}}, \\
&\mathbf{u}_\mathrm{mesh} = 0 & \qquad & \textrm{on } \partial \Omega_{\textrm{Dirichlet}}.
\end{aligned}
\label{eq:laplacian_bcs}
\end{equation}
This scheme has the advantage of working for many different domain geometries and combinations of boundary conditions.
For moderate mesh deformation, the mesh stays smooth and well conditioned, though it breaks down for large deformations 
or non-convex domains.

\subsection{Surface advection}
With a deformable domain there is the danger that small errors in free surface motion can
result in poor overall mass conservation in time. At worst, the total volume of the mesh can 
fluctuate significantly over hundreds or thousands of time steps.
Consistency with the Stokes solution requires 
\begin{equation}
\mathbf{u}_{\mathrm{mesh}} \cdot \mathbf{n} = \mathbf{u \cdot n}
\end{equation}

Unfortunately the normal vectors are not well defined on the mesh vertices, which is 
where the mesh velocity is defined. One could instead advect the mesh in the direction 
of the local vertical, or in some weighted average of the cell normals adjacent to a given vertex,
but we have found that these schemes do not necessarily have excellent mass conservation 
properties.

A better approach is to perform a finite element projection of the Stokes velocity 
solution onto the mesh velocity vector. Multiplying the boundary conditions 
\eqref{eq:laplacian_bcs} by a test function $\mathbf{w}$ and integrating over the free
surface part of the boundary, we find

\begin{equation}
\int_{\partial \Omega_{\textrm{free surface}} } \mathbf{w} \cdot \mathbf{u}_\mathrm{mesh} = 
\int_{\partial \Omega_{\textrm{free surface}} } \left( \mathbf{w \cdot n } \right) \left( \mathbf{u \cdot n} \right)
\end{equation}

When discretized, this forms a linear system which can be solved for the mesh velocity at the 
free surface. Of course, a new system to solve is not ideal, but this system, being nonzero 
over only the free surface, is relatively small and easy to solve.

This weak solution to the boundary conditions~\eqref{eq:laplacian_bcs} is able to borrow
the accuracy of the Stokes solve for Equation~\eqref{eq:weak_incompressible}, and we have 
found it to conserve mass more accurately than algebraic techniques for evaluating the mesh-normal velocity.
Similar results were found by \citet{fullsack1995arbitrary}.

\section{Results}
\begin{itemize}
\item Crameri benchmarks
\item comparison between stabilization schemes (stability)
\item comparison between stabilization schemes (accuracy)
\end{itemize}
\section{Discussion}

%% The Appendices part is started with the command \appendix;
%% appendix sections are then done as normal sections
%% \appendix

%% \section{}
%% \label{}

%% If you have bibdatabase file and want bibtex to generate the
%% bibitems, please use
%%
\bibliographystyle{elsarticle-harv} 
\bibliography{free-surface-paper}

%% else use the following coding to input the bibitems directly in the
%% TeX file.

%\begin{thebibliography}{00}

%% \bibitem[Author(year)]{label}
%% Text of bibliographic item

%\bibitem[ ()]{}

%\end{thebibliography}
\end{document}

\endinput
%%
%% End of file `elsarticle-template-harv.tex'.
