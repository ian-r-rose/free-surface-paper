%% 
%% Copyright 2007, 2008, 2009 Elsevier Ltd
%% 
%% This file is part of the 'Elsarticle Bundle'.
%% ---------------------------------------------
%% 
%% It may be distributed under the conditions of the LaTeX Project Public
%% License, either version 1.2 of this license or (at your option) any
%% later version.  The latest version of this license is in
%%    http://www.latex-project.org/lppl.txt
%% and version 1.2 or later is part of all distributions of LaTeX
%% version 1999/12/01 or later.
%% 
%% The list of all files belonging to the 'Elsarticle Bundle' is
%% given in the file `manifest.txt'.
%% 
%% Template article for Elsevier's document class `elsarticle'
%% with harvard style bibliographic references
%% SP 2008/03/01

\documentclass[preprint,12pt,authoryear]{elsarticle}

%% Use the option review to obtain double line spacing
%% \documentclass[authoryear,preprint,review,12pt]{elsarticle}

%% Use the options 1p,twocolumn; 3p; 3p,twocolumn; 5p; or 5p,twocolumn
%% for a journal layout:
%% \documentclass[final,1p,times,authoryear]{elsarticle}
%% \documentclass[final,1p,times,twocolumn,authoryear]{elsarticle}
%% \documentclass[final,3p,times,authoryear]{elsarticle}
%% \documentclass[final,3p,times,twocolumn,authoryear]{elsarticle}
%% \documentclass[final,5p,times,authoryear]{elsarticle}
%% \documentclass[final,5p,times,twocolumn,authoryear]{elsarticle}

%% For including figures, graphicx.sty has been loaded in
%% elsarticle.cls. If you prefer to use the old commands
%% please give \usepackage{epsfig}

%% The amssymb package provides various useful mathematical symbols
\usepackage{amssymb}
\usepackage{amsmath}
\usepackage{color, soul}
\usepackage{url}
\usepackage{subcaption}
\usepackage{xcolor}
\usepackage{mdframed}


\newif\ifdetail
\detailfalse
%\detailtrue

%% The amsthm package provides extended theorem environments
%% \usepackage{amsthm}

%% The lineno packages adds line numbers. Start line numbering with
%% \begin{linenumbers}, end it with \end{linenumbers}. Or switch it on
%% for the whole article with \linenumbers.
\usepackage{lineno}

%% Block of code for fixing corresponding author bug 
%% in elsarticle template... Don't really understand it
\makeatletter
\def\@author#1{\g@addto@macro\elsauthors{\normalsize%
    \def\baselinestretch{1}%
    \upshape\authorsep#1\unskip\textsuperscript{%
      \ifx\@fnmark\@empty\else\unskip\sep\@fnmark\let\sep=,\fi
      \ifx\@corref\@empty\else\unskip\sep\@corref\let\sep=,\fi
      }%
    \def\authorsep{\unskip,\space}%
    \global\let\@fnmark\@empty
    \global\let\@corref\@empty  %% Added
    \global\let\sep\@empty}%
    \@eadauthor={#1}
}
\makeatother
%% End block of weird code

\journal{Physics of the Earth and Planetary Interiors}

\begin{document}

\begin{frontmatter}

%% Title, authors and addresses

%% use the tnoteref command within \title for footnotes;
%% use the tnotetext command for theassociated footnote;
%% use the fnref command within \author or \address for footnotes;
%% use the fntext command for theassociated footnote;
%% use the corref command within \author for corresponding author footnotes;
%% use the cortext command for theassociated footnote;
%% use the ead command for the email address,
%% and the form \ead[url] for the home page:
%% \title{Title\tnoteref{label1}}
%% \tnotetext[label1]{}
%% \author{Name\corref{cor1}\fnref{label2}}
%% \ead{email address}
%% \ead[url]{home page}
%% \fntext[label2]{}
%% \cortext[cor1]{}
%% \address{Address\fnref{label3}}
%% \fntext[label3]{}

\title{Stability and accuracy of free surface time integration in viscous flows}

%% use optional labels to link authors explicitly to addresses:
%% \author[label1,label2]{}
%% \address[label1]{}
%% \address[label2]{}

\author{Ian Rose\corref{cor1}\fnref{ref1}}
\author{Bruce Buffett\fnref{ref1}}
\author{Timo Heister\fnref{ref2}}

\fntext[ref1]{University of California, Berkeley}
\fntext[ref2]{Clemson University}
\cortext[cor1]{Corresponding author, \url{ian.rose@berkeley.edu}}


\address{}

\begin{abstract}
Geodynamic simulations increasingly rely on models with a true free surface to 
investigate questions of dynamic topography, tectonic deformation, gravity perturbations, and
global mantle convection. However, implementations of free surface boundary conditions 
have proven challenging from a standpoint of accuracy, robustness, and stability.
In particular, time integration of a free surface tends to suffer from a numerical instability
that manifests as sloshing surface motions, also known as the ``drunken sailor'' instability.
This instability severely limits stable timestep sizes to those much smaller than can be used
in geodynamic simulations without a free surface. 
Several schemes have been proposed in the literature to deal with these instabilities.

Here we analyze the problem of creeping viscous flow with a free surface and discuss the 
origin of these instabilities. We demonstrate their cause and how existing stabilization 
schemes work to damp them out.
We also propose a new scheme for removing instabilities from free surface calculations. 
It does not require modifications to the system matrix, nor additional variables, but is instead
an explicit scheme based on nonstandard finite differences.  It relies on a single 
stabilization parameter which may be identified with the smallest relaxation timescale of the
free surface.

Finally, we discuss the implementation of a free surface in the open source, community based
mantle convection software \texttt{ASPECT}.
\end{abstract}

\begin{keyword}
Numerical modeling \sep
Free surface \sep
Mantle convection \sep
Geodynamics
%% keywords here, in the form: keyword \sep keyword

%% PACS codes here, in the form: \PACS code \sep code

%% MSC codes here, in the form: \MSC code \sep code
%% or \MSC[2008] code \sep code (2000 is the default)

\end{keyword}

\end{frontmatter}

\linenumbers

%% main text
\section{Introduction}
\label{sec:intro}

Surface topography in simulations of mantle convection and other geodynamic processes is an important observable,
allowing insights into Earth's internal density structure, rheology, and geoid perturbations \citep[e.g.][]{richards1984geoid, hager1985lower, baumann2014constraining}.
Historically, most simulations have been performed with free-slip boundary conditions at the surface, 
with dynamic topography calculated as a postprocessing step \citep[e.g.][]{jarvis1982mantle, zhong2000role}.

There have been several approaches to treating real free surfaces in geodynamic simulations.
\citet{zhong1996free} introduced a pseudo-free-surface formulation, where the free surface coordinate was
treated as an extra variable that was integrated in time. In this formulation, the free surface 
is approximated on the undeformed Eulerian grid by applying pressure boundary conditions on the reference surface.
The pressure is determined by a first-order Taylor series approximation of the hydrostatic pressure profile 
predicted by the surface topography.

A large number of studies have approximated free surfaces in the interior of the domain by using the 
`sticky air' approximation (\citet{matsumoto1983numerical}; for a comprehensive review see \citet{crameri2012comparison} and references therein). In this approximation there is a low-viscosity, low-density layer in the fluid 
(termed `air', though its viscosity is much higher) above the free surface, effectively decoupling it from the boundary. 
Typically a free-slip boundary condition is used above the sticky air, though an open boundary may be preferable \citep{hillebrand2014using}. 

Finally, one can use a true free surface, where a stress-free boundary condition is applied on the surface of the simulation. 
With this setup there can be flow in and out of the free surface, so it must move in time.
In many cases this is accomplished through a Lagrangian or arbitrary Lagrangain-Eulerian (ALE) scheme (see Section~\ref{sec:implementation}).
\citet{duretz2016free} proposed an alternative scheme based on using an Eulerian staggered finite difference mesh
with a modified stencil at the surface to capture the free surface boundary condition.

A true free surface has mathematical elegance in that the boundary condition of the domain more closely 
matches the boundary conditions which one is trying to model, but it typically requires a deformable 
domain with frequent remeshing to avoid ill-conditioned cells. 
In more complex geodynamic simulations there may be portions of the surface which
correspond to regions below air (i.e. continental crust) and other portions that lie below water (i.e. oceanic crust).
In these cases the density contrast at the surface will vary, requiring variations
in the pressure boundary conditions (though viscous stresses should still be zero).

Recently, the nature of surface boundary conditions have been shown to be important for controlling the 
dynamics of subduction zone modeling.  In a benchmark study \citet{schmeling2008benchmark} performed extensive testing on the effect
of a free surface on the sinking of a slab. They found that the nature of the free surface had a large effect 
on the dynamics of the slab, affecting both the shape and the timing of sinking. Most of the 
participating codes in that benchmark used the sticky air approximation.
They found that the specific properties of the sticky air layer controlled the shape and timing of the slab.
Furthermore, the viscosity averaging scheme for areas of transitional
composition was extremely important, as the simulation of the subducting slab entrained significant amounts of the sticky air.
\citet{crameri2012comparison} conducted comparisons between sticky air and true free surface models, 
demonstrating a range of parameters for sticky air which can mitigate some of the difficulties that it introduces.
A study by \citet{quinquis2011role} found that free surface boundary conditions have a large effect on 
trench migration in a subduction zone, and \citet{crameri2012free} found that a free surface combined 
with a weak crust is important for producing one-sided subduction zones.

All of the approaches to free surface simulations have been subject to an instability which has been 
variously termed a ``sloshing,'' or ``drunken sailor'' instability \citep{kaus2010stabilization, duretz2011discretization, kramer2012implicit}. 
This instability, arising from the large density contrast typical at a free surface (compared with the much smaller 
internal density contrasts), severely limits the maximum stable timestep for free surface computations.
Frequently, the maximum stable timestep is several orders of magnitude smaller than that for an otherwise 
equivalent model with free-slip boundary conditions.

Several studies have attempted to alleviate the timestepping requirements imposed by the drunken sailor 
instability. Since the most expensive part of geodynamic simulations is typically the Stokes solve,
most free surface calculations have preferred to use explicit timestepping methods for the free 
surface. \citet{kaus2010stabilization} proposed a quasi-implicit scheme which modifies the discretized 
Stokes matrix, giving it better stability properties.  \citet{popov2008slim3d}, \citet{kramer2012implicit}, and \citet{furuichi2015implicit}
explored methods for solving the transport of the free surface implicitly.

The paper is organized as follows.
After introducing the problem in Section~\ref{sec:governing},
we derive an approach to analyze 
free surface schemes based on the normal modes in Section~\ref{sec:eigenvalue}
and Section~\ref{sec:timestepping}.
In Section~\ref{sec:kmm} we use this approach to look at the
quasi-implicit stabilization proposed in \citet{kaus2010stabilization}.
Then, we propose a new time marching scheme for free surface computations with 
good stability properties (Section~\ref{sec:newscheme}).
Finally, we describe the implementation of a free surface in the mantle convection software \texttt{ASPECT} in Section~\ref{sec:implementation}, 
before showing some numerical results and benchmarks in Section~\ref{sec:results}. 


\section{Governing equations}
\label{sec:governing}

We begin with the incompressible momentum conservation equations for creeping incompressible flow:
\begin{equation}
\begin{aligned}
\nabla \cdot \mathbf{T} &= - \rho \mathbf{g} \\
\nabla \cdot \mathbf{u} &= 0,
\end{aligned}
\label{eq:momentum_mass_conservation}
\end{equation}
where $\mathbf{u}$ is the fluid velocity, $\rho$ is the fluid density, and $\mathbf{g}$ is the force due to gravity.
$\mathbf{T}$ is the stress tensor for a Newtonian fluid, defined by
\begin{equation}
\mathbf{T} = 2 \eta \varepsilon(\mathbf{u}) - p \mathbf{I},
\label{eq:stress_tensor}
\end{equation}
where $\eta$ is the viscosity, $\varepsilon(\mathbf{u}) = \frac{1}{2}(\nabla \mathbf{u} + (\nabla \mathbf{u} )^T )$
is the strain-rate tensor, and $\mathbf{I}$ is the identity tensor.
Substituting the stress tensor into Equation~\eqref{eq:momentum_mass_conservation} gets the familiar form of the Stokes equation:
\begin{equation}
\nabla \cdot \left( 2 \eta \varepsilon( \mathbf{u} ) \right) - \nabla p = - \rho \mathbf{g}.
\label{eq:stokes}
\end{equation}

For the purposes of this analysis it is useful to define a hydrostatic reference state where the 
velocity $\mathbf{u}$ is zero:
\begin{equation}
\nabla p_0 = \rho_0 \mathbf{g},
\label{eq:hydrostatic_stokes}
\end{equation}
where $p_0$ is the reference hydrostatic pressure and $\rho_0$ is a reference density profile (which may vary with depth).
The total pressure and density can then be decomposed into variations about their 
reference values: $\rho = \rho_0 + \rho^\prime$, $p = p_0 + p^\prime$.
Using this, the hydrostatic reference state (Equation~\eqref{eq:hydrostatic_stokes}) may be subtracted from Equation~\eqref{eq:stokes}.

This gives rise to the following time dependent, coupled system with unknowns $\mathbf{u}(t)$ and $\mathbf{p}(t)$:
\begin{equation}
\begin{aligned}
\nabla \cdot \left( 2 \eta \varepsilon( \mathbf{u} ) \right) - \nabla p^\prime &= - \rho^\prime \mathbf{g} \qquad &\text{in } &\Omega(t)\\
\nabla \cdot \mathbf{u} &= 0  \qquad &\text{in } &\Omega(t)
\end{aligned}
\label{eq:final_system}
\end{equation}
defined on the bounded, moving domain $\Omega(t)\subset \mathbb{R}^d$ with boundary $\partial \Omega = \Gamma_0 \cup \Gamma_F$
split into fixed (Dirichlet) and free surface parts $\Gamma_0$ and $\Gamma_F$, respectively.
The boundary conditions are given by
\begin{equation}
\begin{aligned}
\mathbf{u} &= 0 &\qquad \text{on } &\Gamma_0 \\
\mathbf{T}\cdot \mathbf{n} &= 0 &\qquad \text{on } &\Gamma_F.
\end{aligned}
\end{equation}
For the sake of economy, we neglect inhomogeneous stress boundary conditions, though it would be straightforward to include them.
We also defer discussion of free slip boundary conditions.
The domain at time $t$ is defined by advecting a reference configuration $\Omega_0$ by the domain velocity $\mathbf{u}(t)$
\begin{equation}
 \Omega(t) = \Omega_0 + \int_{t_0}^t \mathbf{u}(t) \text{ d}t.
 \label{eq:domain_evolution}
\end{equation}
We defer the discussion of the velocity of the discretized domain to Section~\ref{sec:remeshing}.
Finally, we denote the displacement at time $t$ by $\zeta = \int_{t_0}^t \mathbf{u}_\mathrm{surface}(t)\cdot \mathbf{n} \text{ d}t$, leading to the evolution
equation
\begin{equation}
\frac{\text{d} \zeta}{\text{d}t} = \mathbf{u \cdot \mathbf{n}} \quad \textrm{on  }  \Gamma_F.
\label{eq:surface_evolution}
\end{equation}



\section{Eigenvalue analysis}
\label{sec:eigenvalue}

In order to better understand the time evolution of the system in Equation~\eqref{eq:final_system},
we will consider the eigenvalues of a simplified homogeneous system where we neglect variations in density ($\rho^\prime = 0$):
\begin{equation}
\begin{aligned}
\nabla \cdot \left( 2 \eta \varepsilon( \mathbf{u} ) \right) - \nabla p^\prime &= 0 \\
\nabla \cdot \mathbf{u} &= 0.
\end{aligned}
\label{eq:homogeneous_stokes}
\end{equation}


We will proceed with this analysis within a finite element framework, though similar arguments should 
work for other discrete methods.
We transform the governing equations into the weak form amenable to finite elements via standard operations \citep[e.g.][]{zienkiewicz1977finite} to get

\begin{equation}
\int_{\Omega(t)} 2 \eta \varepsilon( \mathbf{w} ) \colon \varepsilon( \mathbf{u} ) - \int_{\Omega(t)} p^\prime \nabla \cdot \mathbf{w} 
- \int_{\Gamma_F(t)} \mathbf{w} \cdot \mathbf{T} \cdot \mathbf{n} = 0 
\label{eq:weak_stokes}
\end{equation}
\begin{equation}
\int_{\Omega(t)} q \nabla \cdot \mathbf{u} = 0,
\label{eq:weak_incompressible}
\end{equation}
where $\mathbf{w}$ and $q$ are suitably chosen test functions and the integrals over 
$\Omega(t)$ and $\Gamma_F(t)$ are over the volume of the domain and the free surface, respectively.
Note that since the free surface can move, the shape of the domain is a function of time.

The integral over the surface in Equation~\eqref{eq:weak_stokes} accounts for boundary stresses, 
which should be zero when evaluated on a true free surface.
Rather than analyzing finite deformation to the free surface (a nonlinear problem),
we will make the analytically useful approximation of small deformations about the hydrostatic 
reference surface and analyze their stability.
We will therefore  evaluate the integrals in Equation~\eqref{eq:weak_stokes} 
over the hydrostatic reference surface and introduce a temporary auxiliary variable $\zeta$ which 
represents the (small) topography of the free surface relative to that reference surface.
On the reference surface the gravity vector is opposite the direction of the surface normal $\mathbf{g} = -g \mathbf{n}$.
The stress on this reference surface can be approximated by using the first order Taylor series
expansion of the hydrostatic pressure profile:
\begin{equation}
\mathbf{T} \approx \frac{\partial \mathbf{T}}{\partial \mathbf{n} } \cdot \mathbf{n} \zeta = -\rho_0 g \zeta \mathbf{I}.
\label{eq:hydrostatic}
\end{equation}
Equation~\eqref{eq:weak_stokes} then becomes

\begin{equation}
\int_{\Omega(t)} 2 \eta \varepsilon( \mathbf{w} ) \colon \varepsilon( \mathbf{u} ) - \int_{\Omega(t)} p^\prime \nabla \cdot \mathbf{w} 
+ \int_{\Gamma_F(t)} \rho_0 g \zeta  \mathbf{w} \cdot \mathbf{n} = 0.
\end{equation}

We would like to analyze the time evolution of the normal modes of this system: each mode 
is the relaxation of topography with a characteristic relaxation time.  
We denote the normal modes by $\left[ \mathbf{u}_i, p^\prime_i, \zeta_i \right]^T$, with
relaxation times $\tau_i$, where the subscript corresponds to the $i$th normal mode.

The equations decouple for the normal modes, and Equation~\eqref{eq:surface_evolution} then becomes

\begin{equation}
\frac {\text{d}}{\text{d} t} \zeta_i(\mathbf{x},t) = \frac{\text{d}}{\text{d}t} \zeta_i(\mathbf{x})e^{-t/\tau_i} = -\frac{\zeta_i(\mathbf{x},t)}{\tau_i} = \mathbf{u}_i \cdot \mathbf{n}.
\end{equation}
This can then be used to eliminate $\zeta$ from the Stokes system:
\begin{equation}
\int_{\Omega(t)} 2 \eta \varepsilon( \mathbf{w} ) \colon \varepsilon( \mathbf{u}_i ) - \int_{\Omega(t)} p^\prime_i \nabla \cdot \mathbf{w} 
= \tau_i \int_{\Gamma_F(t)} \rho_0 g (\mathbf{u}_i \cdot \mathbf{n} ) (\mathbf{w} \cdot \mathbf{n}).
\label{eq:weak_eigen}
\end{equation}

Equations~\eqref{eq:weak_stokes},~\eqref{eq:weak_incompressible}, and~\eqref{eq:weak_eigen}
are continuous forms of the governing equations which must be discretized in space and time in order to be solved numerically.
In particular, we must decide at which timestep (or combination of timesteps) we will integrate over the domain $\Omega(t)$.
We denote discrete times by $t^n$, where $n$ is the timestep number, and $t^{n+1} = t^n+\Delta t$.
It is most convenient to integrate over the domain at the current timestep $\Omega^n \equiv \Omega(t^n)$ instead of 
the unknown $\Omega^{n+1} \equiv \Omega(t^{n}+\Delta t)$. This corresponds to an explicit scheme.
However, as we will see, the implicit scheme of integrating over the (unknown) domain at the next timestep $\Omega^{n+1}$ will
give a stable timestepping scheme, at the cost of making the problem nonlinear \citep{furuichi2015implicit}.

When these equations are spatially discretized using finite elements \citep[see, e.g.][]{kronbichler2012high} we get
\begin{equation}
\begin{bmatrix}
A & B^T \\
B & 0 \\
\end{bmatrix}
\begin{bmatrix}
\mathbf{u}^h_i \\
p^h_i
\end{bmatrix}
=
\tau_i
\begin{bmatrix}
M & 0 \\
0 & 0
\end{bmatrix}
\begin{bmatrix}
\mathbf{u}^h_i \\
p^h_i
\end{bmatrix},
\label{eq:generalized_eigenvalue}
\end{equation}
where $\mathbf{u}^h_i$ and $p^h_i$ are finite-dimensional representations of $\mathbf{u}_i$ and $p^\prime_i$,
and the domain integrals are performed over a discrete approximation of the domain shape
(such as a linear finite element basis) which we denote by $\hat{\Omega}$.
$M$ is the discretization of the bilinear form on the right-hand side of Equation~\eqref{eq:weak_eigen}.
The form of Equation~\eqref{eq:generalized_eigenvalue} does not depend on whether we choose $\hat{\Omega}^n$ or $\hat{\Omega}^{n+1}$
for the domain integration.

Equation~\eqref{eq:generalized_eigenvalue} is a generalized eigenvalue problem for the normal modes of the system.
It is rather more difficult to solve than a standard eigenvalue problem because the matrix on the right-hand-side 
is not invertible. It may, however, be transformed into a standard eigenvalue problem.
By defining
\begin{equation}
\begin{aligned}
C &= 
\begin{bmatrix}
A & B^T \\
B & 0 \\
\end{bmatrix} \quad
D &= 
\begin{bmatrix}
M & 0 \\
0 & 0
\end{bmatrix} \quad
\mathbf{y}_i &= 
\begin{bmatrix}
\mathbf{u}^h_i \\
p^h_i
\end{bmatrix} 
\end{aligned}
\label{eq:eigen_substitution}
\end{equation}
and then multiplying both sides by $\tau_i^{-1}C^{-1}$ we get
\begin{equation}
(C^{-1}D)\mathbf{y}_i = \tau_i^{-1} \mathbf{y}_i.
\label{eq:standard_eigenvalue}
\end{equation}

This eigenvalue equation can be solved for the normal modes and relaxation times of the Stokes system with 
a free surface.

\section{Time integration of the free surface}
\label{sec:timestepping}

Armed with the normal modes and relaxation times of the Stokes system, we can write down the
formal solution to Equation~\eqref{eq:surface_evolution}. Let the initial surface topography be 
represented by a linear combination of its normal modes
\begin{equation}
\zeta(\mathbf{x}, t=0) = \displaystyle \sum_i a_i \zeta_i(\mathbf{x}),
\end{equation}
then the time evolution of the free surface is given by
\begin{equation}
\zeta( \mathbf{x}, t) = \displaystyle \sum_i a_i \zeta_i(\mathbf{x}) e^{-t/\tau_i}.
\end{equation}
Of course, most finite element (or finite difference, or finite volume) geodynamic simulations do not resolve 
the solution and surface topography into its normal modes and integrate those separately. 
Indeed, analytical solutions for the normal modes are only known for simple geometries and rheologies.
Instead, one typically obtains a velocity solution and then advect the free surface with the local velocity.
The normal mode solution is instructive, however. 
Each mode has the form of a decay equation with characteristic decay time $\tau_i$.
The decay equation is the archetypical example of a stiff ordinary differential equation.
If we were to numerically integrate this in time with a forward Euler method, we would find the 
timestep criterion for stability \citep[e.g.][]{leveque2007finite} to be
\begin{equation}
\Delta t  \le 2 \tau_{\mathrm{min}},
\label{eq:cfl_euler}
\end{equation}
where $\tau_{\mathrm{min}}$ is this minimum decay time.
The maximum stable timestep is limited by the minimum relaxation timescale.
If a larger timestep than this is taken then those modes will become unstable.
The modes with the smallest relaxation times are usually those with the largest lengthscales \citep{schubert2001mantle}, 
and so it will be those which become unstable first, a phenomenon which has been called 
a ``sloshing'', or ``drunken sailor'' instability \citep{kaus2010stabilization}.

This stability analysis does not rely on the specific choice of spatial discretization,
and so should hold equally well for linear finite elements and higher-order elements.

\section{Analysis of quasi-implicit stabilization}
\label{sec:kmm}

The relaxation timescales for surface topography tend to be considerably shorter than those for 
convection or tectonic deformation, so the stability requirements for a forward Euler scheme
are quite onerous.  On the other hand, an implicit time marching scheme requires solving 
a nonlinear system for the new surface position, or assembling a larger system with surface
topography unknowns \citep[e.g.][]{kramer2012implicit}.

\citet{kaus2010stabilization} proposed a scheme whereby the body forces on the domain are 
evaluated on a prediction of the shape of the domain at a later time.
The continuous weak form of the right-hand-side body forces in is then
\begin{equation}
\mathbf{f}_{\mathrm{body}} = \int_{\Omega^n + \Delta \Omega} \rho \mathbf{w} \cdot \mathbf{g}.
\label{eq:predict}
\end{equation}
We can approximate the shape prediction by time integrating the domain evolution equation (Equation~\eqref{eq:domain_evolution}):
\begin{equation}
  \Delta \Omega \approx \theta \Delta t \mathbf{u},
\end{equation}
where $\theta$ is a free parameter that corresponds to the magnitude of the 
correction, where zero is no stabilization and one is fully (quasi) implicit.

One can approximately expand the integral in Equation~\eqref{eq:predict} using 
Reynold's transport theorem to find

\begin{equation}
\int_{\Omega^n + \Delta \Omega} \rho  \mathbf{w} \cdot \mathbf{g} \approx
\int_{\Omega^n} \rho  \mathbf{w} \cdot \mathbf{g} + \theta \Delta t \int_{\Gamma_F^n} \rho ( \mathbf{w \cdot g}) (\mathbf{u \cdot n} ).
\label{eq:kmm}
\end{equation}
The volume integral is the same as that of the unstabilized problem, but now we obtained an additional surface integral.
It has the form of a velocity-dependent surface stress pushing down on the 
domain, and can be thought of as an artificial viscous damping of the surface.
Since the term depends on the velocity, it enters the Stokes system matrix as a stabilization term.  
Empirically, it has been found to be successful at damping instabilities \citep{kaus2010stabilization, quinquis2011role, duretz2011discretization}.

The surface integral in Equation~\eqref{eq:kmm} is almost the same as that on the right hand side of 
Equation~\eqref{eq:weak_eigen}, which was discretized as the matrix $M$.
By making two approximations we can connect it with the formalism developed in Section~\ref{sec:eigenvalue}.
First, when the current position of the surface is near to the reference surface (i.e., the slope is not too steep)
then gravity is approximately in the direction of the surface normal: $\mathbf{g} \approx -g \mathbf{n}$.
Furthermore, in Equation~\eqref{eq:weak_eigen} we considered response of the surface in a homogeneous system where there were no lateral density variations,
while Equation~\eqref{eq:kmm} uses the total density.
In most geodynamic simulations lateral density variations are small compared to the total density.
For the purposes of stabilization we may approximate $\rho \approx \rho_0$.
Inserting these two approximations, we find that the stabilization term may be written as
\begin{equation}
-\theta \Delta t \int_{\Gamma_F^n} \rho_0 g ( \mathbf{w \cdot n}) (\mathbf{u \cdot n} ),
\label{eq:kmm_stabilization}
\end{equation}
where the integral is now identical to that which discretizes to $M$.

If we discretize the Stokes system with the quasi-implicit stabilization term, we find
a new generalized eigenvalue problem:
\begin{equation}
\begin{bmatrix}
A + \theta \Delta t M & B^T \\
B & 0 \\
\end{bmatrix}
\begin{bmatrix}
\mathbf{u}^h_i \\
p^h_i
\end{bmatrix}
=
\tau^S_i
\begin{bmatrix}
M & 0 \\
0 & 0
\end{bmatrix}
\begin{bmatrix}
\mathbf{u}^h_i \\
p^h_i
\end{bmatrix},
\label{eq:stabilized_generalized_eigenvalue}
\end{equation}
where $\tau^S_i$ indicate the eigenvalues of the stabilized system.
This system may be rearranged:
\begin{equation}
\begin{bmatrix}
A & B^T \\
B & 0 \\
\end{bmatrix}
\begin{bmatrix}
\mathbf{u}^h_i \\
p^h_i
\end{bmatrix}
=
\left(\tau^S_i - \theta \Delta t \right)
\begin{bmatrix}
M & 0 \\
0 & 0
\end{bmatrix}
\begin{bmatrix}
\mathbf{u}^h_i \\
p^h_i
\end{bmatrix}.
\label{eq:rearranged_stabilized_generalized_eigenvalue}
\end{equation}

This is precisely the same generalized eigenvalue problem as Equation~\eqref{eq:generalized_eigenvalue},
so its eigenvalues must be the same.  This allows us to write the eigenvalues of the stabilized problem in
terms of those of the unstabilized problem:
\begin{equation}
\tau^S_i = \tau_i + \theta \Delta t.
\label{eq:spectrum_shift}
\end{equation}

Essentially, the stabilization term lengthens every relaxation time for the system by an amount $\theta \Delta t$.
This correspondingly lengthens the maximum stable timestep for the forward Euler method:
\begin{equation}
\Delta t  \le \frac{2}{1-\theta} \tau_{\mathrm{min}}.
\label{eq:cfl_euler_stabilized}
\end{equation}
Note that as $\theta$ goes to one this scheme should become unconditionally stable,
but nonlinear effects and discretization errors due to finite deformation of the surface could 
prevent that stability.

The lengthening of relaxation times due to the quasi-implicit stabilization has an unequal 
effect on the modes. They are all lengthened by the same amount, which is a much bigger 
fraction of the total relaxation time for the shorter-time modes than the longer-time ones.
Therefore the shorter-time modes are effectively damped more.
This can be seen as an attractive feature of the scheme, since those are the least stable modes,
though it can result in less accurate time-marching of the longest wavelengths of the system.

\subsection{Numerical determination of $\tau_\mathrm{min}$}
\label{sec:numerical_determination}

It can be useful to determine the minimum relaxation time $\tau_{\mathrm{min}}$ for a particular
simulation setup, both to determine an appropriate timestep and as a check on the eigenvalue
analysis above. As discussed in Section~\ref{sec:eigenvalue}, the generalized eigenvalue problem
given by Equation~\eqref{eq:stabilized_generalized_eigenvalue} is difficult to solve because the matrix on
the right-hand-side is not invertible.
However, we may convert Equation~\eqref{eq:stabilized_generalized_eigenvalue} to a standard eigenvalue problem
in the same manner as before. Introducing $C_S$ as the Stokes system matrix with the stabilization term
included, and with $D$ and $\mathbf{y}_i$ the same as in Equation~\eqref{eq:eigen_substitution}, we find
\begin{equation}
(C_S^{-1}D)\mathbf{y}_i = \frac{1}{\tau^S_i} \mathbf{y}_i.
\label{eq:stabilized_standard_eigenvalue}
\end{equation}
Several things are of note about this eigenvalue equation which help in solving it.
The matrix $D$ is a mass matrix which is only nonzero on the free surface,
so it is mostly empty, and, in general, it is not necessary to explicitly form the
matrix in order to apply it.
Second, applying the inverse of $C_S$ amounts to solving the Stokes system, so any
existing geodynamical code can already do this.
Third, it is now an equation for the inverse of $\tau^S_i$, and so solving for the 
minimum relaxation time is now equivalent to finding the \emph{maximum} eigenvalue
of Equation~\eqref{eq:stabilized_standard_eigenvalue}.

All of these factors mean that power iteration (see, e.g., \citet{golub2012matrix}),
which finds the dominant eigenvalue/eigenvector pair, is an attractive option for finding $\tau_\mathrm{min}$.
We have tested the effectiveness of power iteration for finding the minimum relaxation time
on a problem for which there is an analytical solution.
We also use it to confirm Equation~\eqref{eq:spectrum_shift}.
Following \citet{kramer2012implicit} we model the relaxation
of surface topography in a two-dimensional Cartesian box with an isoviscous fluid.
The setup is shown in Figure~\ref{fig:benchmark_setup}.
The eigenvectors of this system are given by infinitesimal sinusoidal perturbations to the free surface
\begin{equation}
\zeta(x,0) = \zeta_0 \cos\left( k x \right),
\label{eq:initial_topography}
\end{equation}
where $k = 2 \pi n / L$, and $n$ is a wavenumber which is an integer multiple of $1/2$.
For an infinitesimal perturbation of this form, the relaxation time $\tau$ is given by
\begin{equation}
\tau = \frac{D k + \sinh(D k) \cosh(D k)}{ \sinh^2 (D k) } \frac{2 k \eta}{\rho g},
\label{eq:analytic_relaxation_time}
\end{equation}
where $D$ is the layer depth, $\eta$ is the viscosity, $\rho$ is the density, and $g$ is the acceleration due to gravity.

\begin{figure*}
\includegraphics[width=0.9\textwidth]{figures/benchmark_setup.pdf}
\caption{Setup for the power iteration test (Section~\ref{sec:numerical_determination}) and for the free surface relaxation benchmark (Section~\ref{sec:topography_relaxation}). A 2D box with an isoviscous fluid has sinusoidal initial topography, with amplitude $\zeta_0$ and wavenumber $n$. The box has depth $D$ and length $L$. For our tests $\rho = \eta = g = D = L = 1$, and $n=1/2$. For the power iteration test there is no initial perturbation ($\zeta_0 = 0.0$), and for the surface relaxation benchmark $\zeta_0 = 0.005$.}
\label{fig:benchmark_setup}
\end{figure*}
The eigenvector with the minimum relaxation time for this system $\tau_{\min}$ corresponds to $n=1/2$ 
(which, as expected, has the form of the fluid sloshing back and forth).
Figure~\ref{fig:perturb_spectrum} shows the results of the power iteration, where colored circles indicate the numerical
solution and black dots show the analytical solution using equation~\eqref{eq:spectrum_shift}.
In most cases the iteration converges to better than $\sim10^{-5}$ relative accuracy in about 20 iterations
(and fewer with a good initial guess).


\begin{figure*}
\includegraphics[width=0.9\textwidth]{figures/perturb_spectrum.pdf}
\caption{The effect of quasi-implicit stabilization on the minimum relaxation time of a fluid in a 2D Cartesian box with a free surface (see Figure~\ref{fig:benchmark_setup}), as well as the result of power iteration to find the minimum relaxation time. The $x$-axis shows the timestep normalized by minimum relaxation time of the unstabilized problem $\tau_\mathrm{\min}$. The $y$-axis shows the stabilized relaxation time $\tau_S$, also normalized by $\tau_\mathrm{min}$. Colored circles show the relaxation times found by numerical solution of Equation~\eqref{eq:stabilized_standard_eigenvalue} with different values of the stabilization parameter $\theta$. Black dots show the relaxation times calculated using Equation~\eqref{eq:spectrum_shift}. The stabilized timescale is lengthened for larger timesteps and for larger values of $\theta$. For this test case, power iteration typically attains better than 1\% accuracy in 5-10 iterations.}
\label{fig:perturb_spectrum}
\end{figure*}


\section{A novel time-integration scheme}
\label{sec:newscheme}

The implementation of such a scheme results in a slightly 
asymmetric matrix, which can be more difficult to solve, requiring either
changes to the solver/preconditioner, or symmetrization of the stabilization term \citep{kaus2010stabilization}.
An alternative would be to construct a time-integration scheme that remains stable at larger timesteps.
The reason that forward Euler integration performs so badly with the decay of topography is that 
at larger timesteps it overshoots its equilibrium position. This overshoot causes it to lurch back 
to the \emph{other} side of equilibrium, overshooting even more.
We would like to construct an explicit scheme that accounts for following properties which we know the system has:
\begin{itemize}
\item In the absence of forcing, topography always decreases.
\item Relaxation of small amplitude topography takes the form of exponential decay.
\item The decay mode with the shortest relaxation time is the least stable.
\end{itemize}

\subsection{Nonstandard finite-differences}

A general expression for the evolution of a free surface from time $t^n$ to time $t^{n+1}$ is 
\begin{equation}
\mathbf{x}(t^{n+1}) = \mathbf{x}(t^n) + \int_{t^{n}}^{t^{n+1}} \mathbf{u}(t) \text{ d}t,
\label{eq:time_integration}
\end{equation}
where $\mathbf{x}$ is the location of the free surface.  If we approximate $\mathbf{u}(t) \approx \mathbf{u}(t^{n})$, 
we of course recover the forward Euler scheme.
However, we can make another choice based on our knowledge of the system behavior. 
We can approximate $\mathbf{u}(t)$ as
\begin{equation}
\mathbf{u}(t) = \mathbf{u}(t^n) e^{-(t-t^n)/\tau^*},
\label{eq:velocity_decay}
\end{equation}
where $\tau^*$ is some as-yet undetermined positive constant which we will refer to as the ``stabilization timescale''.
This form of $\mathbf{u}$ automatically decays in time, and as we shall see, has much better 
stability properties than forward Euler integration.
Using Equation~\eqref{eq:velocity_decay} in Equation~\eqref{eq:time_integration} and integrating, we find

\begin{equation}
\mathbf{x}(t^{n+1}) = \mathbf{x}(t^n) + \mathbf{u}(t^{n}) \tau^* \left(1-e^{-\Delta t/\tau^*} \right).
\label{eq:nsfd}
\end{equation}

The quantity $\tau^*(1-e^{-\Delta t / \tau^*})$ acts as a pseudo-timestep for advecting the free-surface.
Equation~\eqref{eq:nsfd} is what is known as a nonstandard finite-difference model, based on
constructing unusual discrete models for differential equation integration.
The theory has been developed in, among others, a series of papers by
\citet{mickens1994nonstandard, mickens2002nonstandard, mickens2005dynamic}.

\subsection{Stability of the scheme}
The parameter $\tau^*$ sets how quickly $\mathbf{u}$ decays in Equation~\eqref{eq:velocity_decay}, and a good 
choice is crucial for accuracy and stability. A shorter decay time corresponds to more stabilization,
but if it becomes too short, the surface velocity will become overdamped. In general, we will want 
to choose $\tau^*$ so that it is as close as possible to the relaxation time of the least stable mode, or $\tau_{\mathrm{min}}$.

To investigate the stability of this scheme we consider a velocity solution comprised of 
a single normal mode of the system $\mathbf{u} = a_i \mathbf{u}_i$.
This will decay exponentially with relaxation time $\tau_i$, or
\begin{equation}
\frac{\text{d} a_i} {\text{d}t} = - \frac{ a_i }{\tau_i}.
\end{equation}
Applying the nonstandard finite difference scheme, we find
\begin{equation}
a_i^{n+1} = a_i^{n} \left[ 1 - \frac{\tau^*}{\tau_i} \left(1-e^{-\Delta t/\tau^*} \right) \right].
\label{eq:recursion}
\end{equation}
In order for the scheme to be stable, the quantity in brackets must remain bounded as it is repeatedly multiplied by itself, or 
\begin{equation}
\left| 1 - \frac{\tau^*}{\tau_i} \left(1-e^{-\Delta t/\tau^*} \right) \right| \le 1.
\end{equation}

\ifdetail
Taking the positive value of the absolute value yields the criterion of 
\begin{equation}
- \frac{\tau^*}{\tau_i} \left(1-e^{-\Delta t/\tau^*} \right) \le 0,
\end{equation}
which can be rearranged to find
\begin{equation}
e^{-\Delta t/\tau^*} \le 1,
\end{equation}
which is simply a statement that the timestep must be positive.

Taking the negative value of the absolute value yields
\begin{equation}
  \frac{\tau^*}{\tau_i} \left(1-e^{-\Delta t/\tau^*} \right)  \le 2
\end{equation}
\begin{equation}
 1 - 2 \frac{\tau_i}{\tau^*} \le e^{-\Delta t/\tau^*}
\end{equation}
if the left hand side is less than zero, then this is true regardless of step size, or
\begin{equation}
\tau^* \le 2 \tau_i
\end{equation}
if the left hand side is between zero and one, the expression is more complicated. 
Taking the log of both sides:
\begin{equation}
\Delta t \le -\tau^* \log \left(1 - 2 \frac{\tau_i}{\tau^*} \right).
\end{equation}
It is convenient to write this in terms of dimensionless times $\Delta t/ \tau_i$ and $\tau^*/\tau_i$:
\begin{equation}
\frac{\Delta t}{\tau_i} \le -\frac{\tau^*}{\tau_i} \log \left(1 - 2 \frac{\tau_i}{\tau^*} \right).
\end{equation}

\fi

The stability of this scheme is determined by the choice of $\tau^*$.  
It has a region of unconditional stability, where
\begin{equation}
\tau^* \le 2 \tau_i.
\label{eq:nsfd_unconditional}
\end{equation}
Otherwise, the scheme is conditionally stable, with the timestep limited by 
\begin{equation}
\Delta t \le -\tau^* \log \left(1 - 2 \frac{\tau_i}{\tau^*} \right).
\label{eq:nsfd_conditional}
\end{equation}
The stability region is plotted in Figure~\ref{fig:stability_region}. 

\begin{figure*}
\includegraphics[width=0.9\textwidth]{figures/stability_region.pdf}
\caption{(a) Stability region for the nonstandard finite difference scheme (green, Equations~\eqref{eq:nsfd_unconditional} and~\eqref{eq:nsfd_conditional}). On the x-axis is the value of the stabilization timescale, in units of the minimum relaxation time $\tau_{\mathrm{min}}$.  On the y-axis is the value of the timestep, also in units of the minimum relaxation time. For $\tau^*\le 2 \tau_{\mathrm{min}}$ the nonstandard finite difference scheme is unconditionally stable.  (b) Stability region for the quasi-implicit scheme (green, Equation~\eqref{eq:cfl_euler_stabilized}).  Again, the y axis is in units of the minimum relaxation time.  The x-axis shows the value of the stabilization parameter $\theta$. For $\theta = 1$ the quasi-implicit scheme is unconditionally stable. In both cases the stability region for the forward Euler scheme (Equation~\eqref{eq:cfl_euler}) is also shown in blue.}
\label{fig:stability_region}
\end{figure*}

\subsection{Accuracy and asymptotics of the scheme}
It is important to note that even though we derived the nonstandard finite difference scheme assuming a decaying exponential
for $\mathbf{u}$, it is formally a first-order accurate scheme. As such, it is capable of capturing arbitrary motions 
of the free surface, while allowing larger timesteps than forward Euler schemes.

Again we may take one of the normal modes as an example and inspect the difference between the nonstandard finite-difference
scheme and the analytical solution after one timestep:

\begin{equation}
\begin{aligned}
a_i(\Delta t) - a_i^{1} &= a_i(0) e^{-\Delta t/\tau_i} - a_i{(0)} \left[ 1 - \frac{\tau^*}{\tau_i} \left(1-e^{-\Delta t/\tau^*} \right) \right] \\
                        &= a_i{(0)} \left[ e^{-\Delta t/\tau_i} - 1 + \frac{\tau^*}{\tau_i} \left(1-e^{-\Delta t/\tau^*} \right) \right].
\end{aligned}
\end{equation}
Note that when $\tau_i = \tau^*$ the nonstandard finite difference scheme is exact.
The exponentials may be expanded to find
\ifdetail
\begin{equation}
a_i(\Delta t) - a_i^{1} = - a_i{(0)} \left[ \left(\frac{\Delta t}{\tau_i}\right)^2 - \frac{\tau^*}{\tau_i} \left(\frac{\Delta t }{\tau^* }\right)^2 \right].
\end{equation}
\fi
\begin{equation}
a_i(\Delta t) - a_i^{1} = {a_i{(0)} } \left( \frac{\Delta t}{\tau_i} \right)^2 \left( 1 - \frac{\tau_i}{\tau^*} \right) + O(\Delta t^3).
\end{equation}
The local truncation error is quadratic in $\Delta t$, similar to the forward Euler, and summing this error over 
many timesteps results in a factor of $\Delta t^{-1}$, demonstrating the first-order global accuracy~\citep[e.g.][]{leveque2007finite}.
The choice of $\tau^*$ controls the size of the coefficient on the truncation error for the scheme.
The error for a given mode becomes considerably smaller when $\tau^*$ is close to its natural relaxation times.

It is helpful to take a closer look at the pseudo-timestep introduced in Equation~\eqref{eq:nsfd}: $\tau^*(1-e^{-\Delta t/\tau^*})$.
As the timestep $\Delta t$ goes to zero, the pseudo timestep approaches $\Delta t$, recovering 
the forward Euler scheme. However, as $\Delta t$ gets larger, the pseudo-timestep does not 
increase as quickly, reflecting the decaying nature of the normal modes
(in fact, the pseudo-timestep is bounded between $\Delta t$ and $\tau^*$).
Likewise, as the stabilization timescale $\tau^*$ goes to infinity, we also recover the 
forward Euler scheme, in what is essentially the unstabilized problem. But as $\tau^*$
goes to zero, the pseudo-timestep also goes to zero. This corresponds to over-stabilizing the
problem. With too short of a stabilization timescale the free surface is never advected 
at all (which is a very stable situation, if not very accurate!).

\subsection{Choice of $\tau^*$}
\label{sec:tau_choice}
As discussed above, a full geodynamic simulation will have a spectrum of relaxation times.
For complete stability, the parameter $\tau^*$ should be chosen such that every mode is stable.
In practice, this means that a good choice is 
\begin{equation}
\tau^* \approx \tau_{\mathrm{min}}.
\label{eq:tau_choice}
\end{equation}

Unfortunately, for many models the minimum relaxation time will not be known beforehand. 
In order to use the nonstandard finite difference scheme, the value of $\tau^*$ would need 
to be calculated or estimated first.  There are several possible ways to determine this value:

\begin{itemize}
\item Direct solution of Equation~\eqref{eq:generalized_eigenvalue}. Solution of the whole spectrum is expensive
when only the minimum relaxation time is required, which means that power iteration on 
the standard eigenvalue problem (Equation~\eqref{eq:standard_eigenvalue}) can be enough.
\item Analytical formulae. Several geometry and viscosity model combinations have analytical solutions
for relaxation spectra. Even if the user's model is not precisely the same (e.g., has some lateral viscosity
variations), an analytical approximation may be sufficient.
\item Scaling. In general, we expect the relaxation times to scale with $\tau \sim \frac{\eta}{\rho g L}$,
where the density, gravity, viscosity, and lengthscale $L$ are all representative values.  A small amount
of experimentation near this value of $\tau$ can find an appropriate relaxation time.
\end{itemize} 

The first point deserves a bit more attention, since it provides the most general method for determining $\tau_\mathrm{min}$.
We discussed the numerical determination of $\tau_\mathrm{min}$ in Section~\ref{sec:numerical_determination}
by solving Equation~\eqref{eq:standard_eigenvalue} via power iteration.
Each iteration requires a solution of the Stokes system (the $C$ matrix), and we have found
that $\sim$10 iterations is usually enough to obtain a reasonable estimate of $\tau_\mathrm{min}$.
This can be done during preprocessing, so over the course of the simulation the cost of determining
$\tau_\mathrm{min}$ would be negligible.

As a simulation progresses, there is the possibility that its viscosity or density structure
will change, and as such its minimum relaxation time will evolve.
This, of course, means that the best choice of $\tau^*$ can change over the course of a simulation.
If $\tau_\mathrm{min}$ does not change significantly, the initial estimate can be fine.
Otherwise, one can periodically check the evolution and redetermine $\tau_\mathrm{min}$ by 
solving Equation~\eqref{eq:standard_eigenvalue} during the model run.

In Section~\ref{sec:results} we show examples the numerical solution of $\tau_\mathrm{min}$
and its evolution in time.

\label{sec:dynamic_relaxation}

\section{Implementation in \texttt{ASPECT} }
\label{sec:implementation}
We have implemented the ability to run free surface simulations in the mantle convection software \texttt{ASPECT} \citep{aspectweb,kronbichler2012high}. 
\texttt{ASPECT}, based on the open source, finite element library deal.II \citep{dealII82}, is designed to be highly flexible and modular, with the ability for user-defined rheologies, geometries, and gravity models. 
The implementation of the free surface, therefore, needs to be general enough to work for many combinations of these models, including those 
which may not have been written yet. In particular, it cannot rely on assumptions regarding the shape of the domain.

Furthermore, \texttt{ASPECT} allows for computations in both 2D and 3D, so the implementation must be sufficiently dimension
independent to work in both cases. We implement the free surface within an arbitrary Lagrangian-Eulerian (ALE) formulation
\citep[e.g][]{fullsack1995arbitrary,donea2004encyclopedia}.

\texttt{ASPECT} is a parallelized, distributed memory code with adaptive mesh refinement (see \cite{BBHK10} for details).
The free surface implementation works with these features.  We store the mesh vertex positions 
in a fully distributed vector. This vector is continually updated and redistributed across 
processes upon mesh adaptation (which is handled by the adaptive octree library \texttt{p4est}).
We also provide adaptive refinement indicators based on being near to the free surface or 
when the free surface slope is steep to allow for accurate interface tracking.


\subsection{Remeshing}
\label{sec:remeshing}

The mesh velocity in normal direction at the free surface (with unit normal $\mathbf{n}$) has to be
consistent with the velocity of the Stokes velocity solution $\mathbf{u}(t)$:
\begin{equation}
 \mathbf{u}_\mathrm{mesh}(t) \cdot \mathbf{n} = \mathbf{u}(t) \cdot \mathbf{n} \quad \text{on} \quad \Gamma_F
\end{equation}
In ALE calculations the internal mesh velocity is undetermined.
In general, one wants to smoothly deform the mesh so as to preserve its regularity, 
avoiding inverted or otherwise poorly conditioned cells.
The mesh deformation can be calculated in many different ways, including algebraic \citep[e.g.][]{thieulot2011fantom} 
and PDE based approaches.

We choose to implement remeshing based on solving Laplace's equation for the mesh velocity.
We solve the equation
\begin{equation}
\nabla^2 \mathbf{u}_{\mathrm{mesh}} = 0
\label{eq:laplacian_smoothing}
\end{equation}
subject to the boundary conditions
\begin{equation}
\begin{aligned}
&\mathbf{u}_\mathrm{mesh} &= 0  &\qquad \textrm{on } \Gamma_0. \\
&\mathbf{u}_\mathrm{mesh} &= \left( \textbf{u} \cdot \textbf{n} \right) \textbf{n} & \qquad \textrm{on } \Gamma_F, \\
&\mathbf{u}_\mathrm{mesh} \cdot \textbf{n} &= 0  &\qquad \textrm{on } \Gamma_{FS}, \\
\end{aligned}
\label{eq:laplacian_bcs}
\end{equation}
where $\Gamma_{FS}$ is the part of the boundary with free slip boundary conditions, given by
\begin{equation}
\begin{aligned}
 \mathbf{u \cdot n} &= 0 \\ %&\qquad \textrm{on } \Gamma_{FS} \\
 \mathbf{T}\cdot \mathbf{n} - (\mathbf{n}\cdot\mathbf{T} \cdot \mathbf{n})\mathbf{n} &= 0 &\qquad \textrm{on } \Gamma_{FS}.
\end{aligned}
\label{eq:free_slip_bcs}
\end{equation}
This scheme has the advantage of working for many different domain geometries and combinations of boundary conditions.
For moderate mesh deformation, the mesh stays smooth and well conditioned, though it breaks down for large deformations.

\subsection{Surface advection}
With a deformable domain there is the danger that small errors in free surface motion can
result in poor overall mass conservation in time. In some scenarios, the total volume of the mesh can 
fluctuate significantly over hundreds or thousands of timesteps.
Consistency with the Stokes solution requires 
\begin{equation}
\mathbf{u}_{\mathrm{mesh}} \cdot \mathbf{n} = \mathbf{u \cdot n}.
\end{equation}

Unfortunately the normal vectors are not well defined on the mesh vertices, which is 
where the mesh velocity is defined. One could instead advect the mesh in the direction 
of the local vertical, or in some weighted average of the cell normals adjacent to a given vertex,
but we have found that these schemes do not necessarily have good mass conservation 
properties.

A better approach is to perform an $L_2$ projection of the normal velocity $\mathbf{u}\cdot\mathbf{n}$
onto the free surface $\Gamma_F$. Multiplying the boundary conditions 
(Equation~\eqref{eq:laplacian_bcs}) by a test function $\mathbf{w}$ and integrating over the free
surface part of the boundary, we find

\begin{equation}
\int_{\Gamma_F} \mathbf{w} \cdot \mathbf{u}_\mathrm{mesh} = 
\int_{\Gamma_F} \left( \mathbf{w \cdot n } \right) \left( \mathbf{u \cdot n} \right).
  \label{eq:mesh_velocity_system}
\end{equation}
When discretized, this forms a linear system which can be solved for the mesh velocity $\mathbf{u}_\mathrm{mesh}$ at the 
free surface. A new system to solve is not ideal, but this system, being nonzero 
over only the free surface, is relatively computationally inexpensive to solve.

This weak solution to the boundary conditions (Equation~\eqref{eq:laplacian_bcs}) is able to borrow
the accuracy of the Stokes solve for Equation~\eqref{eq:weak_incompressible}, and we have 
found it to conserve mass more accurately than algebraic techniques for evaluating the mesh-normal velocity.
Similar results were found by \citet{fullsack1995arbitrary}.

Figure~\ref{fig:pseudocode} shows pseudocode descriptions of the free surface implementation described in this section.

\begin{figure}
\centering
\begin{subfigure}[b]{0.43\textwidth}
\begin{mdframed}[backgroundcolor=gray!20]
{\small
\begin{enumerate}
  \item Solve the Stokes system, including the boundary term given in Equation~\eqref{eq:kmm}, 
    which is a function of timestep $t^n$ and $\theta$.
  \item Solve for the surface mesh velocity using Equation~\eqref{eq:mesh_velocity_system}.
  \item Solve for the internal mesh velocity using Equations~\eqref{eq:laplacian_smoothing}~-~\eqref{eq:laplacian_bcs}.
  \item Advect the mesh forward in time using displacements determined by the forward Euler scheme: $\mathbf{x}(t^{n+1}) = \mathbf{x}(t^n) + \mathbf{u}_\mathrm{mesh}\Delta t$.
\end{enumerate} 
}
\end{mdframed}
\caption{Quasi-implicit}
\end{subfigure}
\begin{subfigure}[b]{0.43\textwidth}
\begin{mdframed}[backgroundcolor=gray!20]
{\small
\begin{enumerate}
  \item Estimate $\tau_\mathrm{min}$ using one of the means described in Section~\ref{sec:tau_choice}.
  \item Solve the Stokes system normally.
  \item Solve for the surface mesh velocity using Equation~\eqref{eq:mesh_velocity_system}.
  \item Solve for the internal mesh velocity using Equations~\eqref{eq:laplacian_smoothing}~-~\eqref{eq:laplacian_bcs}.
  \item Advect the mesh forward in time using displacements determined by the nonstandard finite difference scheme (Equation~\eqref{eq:nsfd}).
\end{enumerate}
\vspace{1.4mm}
}
\end{mdframed}
\caption{Nonstandard finite differences}
\end{subfigure}
\caption{Pseudocode descriptions of the our free surface implementations.}
\label{fig:pseudocode}
\end{figure}


\section{Numerical results}
\label{sec:results}

\subsection{Relaxation of sinusoidal topography}
\label{sec:topography_relaxation}

We demonstrate the convergence of the nonstandard finite difference model by comparison 
with an analytical solution. 
As in Section~\ref{sec:numerical_determination} we consider the relaxation
of sinusoidal surface topography in a two-dimensional Cartesian box with an isoviscous fluid.
The setup is shown in Figure~\ref{fig:benchmark_setup}.

For an initial topography given by Equation~\eqref{eq:initial_topography}, the relaxation time $\tau$
is given by Equation~\eqref{eq:analytic_relaxation_time}, and the time evolution of the surface 
topography is given by
\begin{equation}
  \zeta(x,t) = \zeta(x,0) e^{-t/\tau}.
\end{equation}
This solution is only valid for infinitesimal topography. However, for small
initial topography $\zeta_0$ it seems to be sufficiently accurate to test convergence orders up 
to quadratic \citep{kramer2012implicit, furuichi2015implicit}.
We estimate the error $E$ by time-integrating the $L_2$ difference between the numerical and analytical
solutions at the center point of the free surface, over a time interval $T$ (which we choose to be $4\tau$):
\begin{equation}
E = \frac{1}{T}\int_0^{T} \lVert \zeta_\mathrm{numeric}(L/2, t) - \zeta_\mathrm{analytic}(L/2, t) \rVert_2 \text{ d}t.
\end{equation}

Figure~\ref{fig:timestep_convergence}a shows the convergence of the nonstandard finite difference scheme 
with respect to timestep size. The scheme is first order in time, with improving accuracy as the value of $\tau^*$ approaches
the relaxation time of the relevant mode $\tau_i$. Interestingly, if $\tau^* = \tau_i$ the advection scheme
becomes exact \citep{mickens2002nonstandard}. At this point the magnitude of the error plummets and is no longer 
a strong function of the timestep. The remaining error is likely due to error in the linear approximation for 
the analytical solution, the spatial discretization, or the linear solver tolerance. 

Figure~\ref{fig:timestep_convergence}b shows the convergence of the quasi-implicit scheme as a function of timestep $\Delta t$.
When $\theta=0$, it corresponds to the unstabilized forward Euler scheme, and is first order in time. 
When $\theta=1$ it is also first order in time, but allows for a much larger timestep.  When $\theta=0.5$
the quasi-implicit guess for the body force is good enough that it actually seems to achieve the second-order 
convergence of a trapezoidal scheme, though it has not been shown that this extends to more complicated models.

Figure~\ref{fig:tau_sensitivity} shows in more detail the effects of the choice of $\tau^*$. In a narrow region 
close to the true value of $\tau_\mathrm{min}$ the error becomes very small, but in a broader region nearby 
the errors are larger. 
The excellent accuracy when the stabilization timescale $\tau^*$ is close to one of the natural relaxation
timescales allows for tuning of the scheme to track specific long-wavelength modes, such as 
those due to rotational or tidal deformation.

\begin{figure*}
\includegraphics[width=0.95\textwidth]{figures/timestep_convergence.pdf}
\caption{Convergence tests for the benchmark shown in Figure~\ref{fig:benchmark_setup}. (a) Convergence test with timestep size for the nonstandard finite difference scheme. Comparison with the slope-one line confirms that it is first-order in time. In the case that the stabilization timescale $\tau^*$ is equal to the analytic relaxation time the error becomes very small, as the time integration scheme becomes exact \citep{mickens2002nonstandard}. (b) Convergence test with timestep size for the quasi-implicit scheme. For $\theta = 1$ and $\theta = 0$ the scheme is first-order accurate (though the latter is just an unstabilized forward-Euler scheme). For $\theta=0.5$ the scheme appears second order accurate on this benchmark.}
\label{fig:timestep_convergence}
\end{figure*}

\begin{figure*}
\includegraphics[width=0.9\textwidth]{figures/tau_sensitivity.pdf}
\caption{Sensitivity of the nonstandard finite difference scheme to $\tau^*$. As the relaxation parameter $\tau^*$ approaches the relaxation time of the benchmark case the error reduces. The sharp cusp at $\tau^* = \tau_\mathrm{min}$ corresponds to the almost spectral accuracy of the time marching scheme for that case.}
\label{fig:tau_sensitivity}
\end{figure*}

\subsection{Rayleigh-Taylor test}
\label{sec:rayleigh_taylor}

\begin{figure*}
\includegraphics[width=0.9\textwidth]{figures/rayleigh_taylor_setup.pdf}
\caption{Setup for the Rayleigh-Taylor test. A denser, more viscous layer of lies on top of the mantle. The overlying layer has a thickness of 100 km, and a sinusoidal perturbation on its base with and amplitude of 5 km.}
\label{fig:rayleigh_taylor_setup}
\end{figure*}

We also consider the time evolution of the Rayleigh-Taylor benchmark described in \citet{kaus2010stabilization}.
In this benchmark a dense, viscous layer with a sinusoidal initial perturbation drips into a less dense, less viscous mantle.
The benchmark setup is shown in Figure~\ref{fig:rayleigh_taylor_setup}.
In all cases we use adaptive mesh refinement with five global refinement levels and four adaptive refinement levels,
corresponding to $\sim$1 km cells at the most refined level. We refine according to compositional gradients, as well as
near to the free surface.

Figure~\ref{fig:rayleigh_taylor} shows the maximum depth of the interface between the dense and light fluid through time.
The blue line shows the results using forward Euler timestepping with a very small timestep of 500 yrs, and is used as a reference solution.
Nonstandard finite difference timestepping and quasi-implicit timestepping both converge to the reference solution,
though from different directions.
For the nonstandard finite difference simulations we recomputed the value of the minimum relaxation time $\tau_\mathrm{min}$ 
every 50 timesteps using power iteration, and reset the value of $\tau^*$ to $\tau_\mathrm{min}$.

The behavior of the two schemes is similar, though at the same CFL number quasi-implicit timestepping is slightly more accurate.
On the other hand, with our implementation we have found that nonstandard finite difference timestepping can allow for larger timesteps 
than quasi-implicit timestepping while remaining stable.

\begin{figure*}
\includegraphics[width=0.9\textwidth]{figures/rayleigh_taylor.pdf}
\caption{Rayleigh-Taylor test. (a) Maximum depth of the drip. The blue line is the result using forward Euler (FE) integration of the free surface with a small 500 yr timestep, and is used as a reference solution. The symbols are for several different runs with different timestepping parameters, using both quasi-implicit (QI) and nonstandard finite difference (NSFD) schemes. As the timestepping gets smaller, both schemes converge to the reference solution (though from different directions). (b) Result of the small timestep forward Euler simulation after 5 Myr, including the adaptive mesh.}
\label{fig:rayleigh_taylor}
\end{figure*}

We also used the Rayleigh-Taylor benchmark to investigate the effect of making a poor choice for $\tau^*$, shown in Figure~\ref{fig:rayleigh_taylor_tau_choice}.
We include four cases, all using adaptive timestepping with CFL=0.2.
As a reference solution we include the case where we update $\tau^*$ to be equal to $\tau_\mathrm{min}$ every 50 timesteps,
allowing for time variations in the viscosity and density structure to change the minimum relaxation time.
We also show the case where we do not update $\tau^*$ over the course of the simulation, and two cases where
we use the wrong value by a factor of two in either direction.

Over the course of the Rayleigh-Taylor benchmark, $\tau_\mathrm{min}$ varies by about 25\%, as shown in Figure~\ref{fig:rayleigh_taylor_tau_choice}b.
The case where we use the initial value of $\tau_\mathrm{min}$ for $\tau^*$ gives nearly identical results,
whereas the cases where we choose values $\sim$50\% off in either direction give significantly different (though not wildly so) results.

In many simulations we expect that the geometry, density, and viscosity structure of the model will not change
enough that the $\tau^*$ would need to be updated frequently (if ever).

\begin{figure*}
\includegraphics[width=0.9\textwidth]{figures/rayleigh_taylor_tau_choice.pdf}
\caption{Effect of the choice of $\tau^*$ on the Rayleigh-Taylor test. (a) Maximum depth of the drip. In all cases adaptive timestepping is used with CFL=0.2. The blue line serves as a reference solution, and shows the result when the smallest relaxation timescale $\tau_{\mathrm{min}}$ is recalculated every 50 timesteps, and then the stabilization timescale $\tau^*$ is set to that value. The red hexagons show the result when we do not update $\tau^*$, instead keeping it at the initial value of $\tau_{\mathrm{min}}$. The results are indistinguishable from those of the first case. The triangles show the results when we make a poor choice of $\tau^*$, by a factor of two in either direction. In these cases there is a significant difference from the reference solution. (b) Evolution of the minimum relaxation timescale over the course of the simulation. It does experience significant changes, but not by orders of magnitude, so it is unsurprising that using the initial value is sufficient for this case.}
\label{fig:rayleigh_taylor_tau_choice}
\end{figure*}


\subsection{Mesh adaptivity}
\label{sec:mesh_adaptivity}

We also investigate the advantages of combining adaptive mesh refinement with free surface computations.
\citet{crameri2012comparison} performed a community benchmark of a setup for which there is no analytic
solution. In this benchmark, a buoyant blob rises beneath a viscous lid with a free surface, deflecting the boundary upwards.
Figure~\ref{fig:amr} shows the convergence of the maximum topography at 3 Myr to its value in a high resolution simulation,
both with and without adaptive mesh refinement.

For the uniform refinement cases each point is generated by running the benchmark with a different global refinement level.
For the adaptive case each point is generated by allowing the mesh to be refined according to gradients in density and composition,
where the maximum refinement level is limited to the same refinement level of the corresponding global refinement simulation.
In both the uniform and the adaptive cases, the smallest cell size in the most refined case is $\sim$1 km.
We refine every ten timesteps according to gradients in the density and compositional fields.

We run simulations using both quasi-implicit and nonstandard finite difference schemes for the free surface.
We use a very small timestep for this test of 500 yr (note that this is significantly smaller than the smallest
relaxation time of $\sim$ 14 kyr), so most timestepping schemes should be stable and accurate, including forward Euler.
We choose this step for two reasons: (1) The primary purpose of this test is to investigate the savings
due to using adaptive mesh refinement with a free surface, and (2) \citet{crameri2012comparison}
found that a timestep of several hundred years was required to achieve full convergence.

The convergence with and without adaptive mesh refinement have essentially the same behavior, but the adaptive case requires 
fewer degrees of freedom by a factor of approximately an order of magnitude. More complex models will have
more detail and so we may be less able to aggressively coarsen them, but we still expect that adaptive mesh refinement 
will result in significant computational savings.

\begin{figure*}
\includegraphics[width=0.9\textwidth]{figures/amr.pdf}
\caption{Convergence with degrees of freedom (DoFs) for the \citet{crameri2012comparison} Case 2 benchmark, for both uniform and adaptive mesh refinement. The timestep $\Delta t$ is 500 yr. We compare the maximum topography at 3 Myr with its value at high resolution ($\sim$398 m). For the adaptive cases we perform mesh refinement according to the sum of density and composition gradients. Quasi-implicit (QI) and nonstandard finite difference (NSFD) timestepping converge similarly, but the adaptive mesh refinement simulations require significantly fewer unknowns. For NSFD timestepping we used a $\tau^*$ of 14.825 kyr, as determined by an analytical solution of the long-wavelength relaxation time of the free surface for this particular model setup \citep{crameri2012comparison}.}
\label{fig:amr}
\end{figure*}

\section{Conclusion}
We have analyzed stability of free surface boundary conditions in geodynamic simulations and 
demonstrated the cause of the drunken sailor instability using a normal mode analysis.
This perspective on the problem allowed us to construct an explicit finite difference 
scheme which is first order accurate in time and is unconditionally stable.
The nonstandard finite difference scheme is simple to implement, and 
requires no modifications to the system matrix.

The normal mode perspective on the problem also provides insights into the effect of 
the quasi-implicit stabilization scheme proposed by \citet{kaus2010stabilization}.
The relaxation time of each mode is lengthened by an amount $\theta \Delta t$, and the
maximum allowable timestep is correspondingly lengthened.

It is not clear that the non-standard finite difference scheme is superior to 
the quasi-implicit scheme. For $\theta=0.5$ the quasi-implicit scheme is more accurate,
but has a smaller stability region. For $\theta=1$ the two schemes are comparably accurate,
with a slight advantage to the quasi-implicit scheme. However, we have found that
the nonstandard finite difference scheme allows for larger stable timesteps, if the modeler 
is willing to pay the price of reduced accuracy for a particular simulation.

Finally, we have described the implementation of free surface boundary conditions in 
the open source mantle convection software \texttt{ASPECT}. Both the quasi-implicit 
scheme and the nonstandard finite difference scheme are available. The implementation is 
sufficiently general to accommodate many different geometries, rheologies, and 
gravity models. Furthermore, it runs in parallel and with adaptive mesh refinement.


\section*{Acknowledgements}
This work was supported through National Science Foundation grant EAR-1246670
and by the Computational Infrastructure in Geodynamics (CIG), through National Science Foundation grant EAR-0949446. 
TH is partially supported by NSF grant DMS-1522191.
Figures were created using the Python packages NumPy and Matplotlib.
Thanks to Wolfgang Bangerth, Brent Delbridge, and Cedric Thieulot for useful discussions.
Thanks to Fabio Crameri and Dave May for constructive reviews.

%% The Appendices part is started with the command \appendix;
%% appendix sections are then done as normal sections
%% \appendix

%% \section{}
%% \label{}

%% If you have bibdatabase file and want bibtex to generate the
%% bibitems, please use
%%

\nocite{hunter2007matplotlib}
\nocite{vanderwalt2011numpy}

\section*{Bibliography}
\bibliographystyle{elsarticle-harv} 
\bibliography{free-surface-paper}

%% else use the following coding to input the bibitems directly in the
%% TeX file.

%\begin{thebibliography}{00}

%% \bibitem[Author(year)]{label}
%% Text of bibliographic item

%\bibitem[ ()]{}

%\end{thebibliography}
\end{document}

\endinput
%%
%% End of file `elsarticle-template-harv.tex'.
