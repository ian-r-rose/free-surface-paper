\documentclass[a4paper,12pt]{article}
\usepackage[utf8]{inputenc}
\usepackage{csquotes}

%opening
\title{Response to referees}
\author{Ian Rose, Bruce Buffett, and Timo Heister}
\date{}

\begin{document}

\maketitle

We thank Taras Gerya and the anonymous referee for their thoughtful and constructive reviews,
to which we respond inline here.

\section*{ Reviewer 1}

\begin{quotation}
  This paper deals with the (now) notorious issue of the drunken-sailor numerical instability that plagues free-surface calculations particularly where the surface deformation is the primary driver of flow and the flow rates are decaying. The same instability also occurs for stable internal boundaries where there same dynamic arguments apply.
\end{quotation}

\begin{quotation}
There have been a number of mechanisms suggested for "fixing" this instability published in recent years and this MS adds a new take to each of those (and appropriately acknowledges / contextualises them). I think this work is a significant addition to the collection of methods available and I would recommend that the MS be published. There has already been significant to-and-fro in review and this is apparent from the quality of the writing and clarity of the explanation. Nevertheless, every fresh pair of eyes brings some new suggestions and I would make the following comments (which I don't consider binding):
\end{quotation}



\begin{quotation}
I find it helpful to have a pseudo-code explanation of any implementation. There is a lot of derivation but in the end it appears that we have a relatively simple algorithm which pulls together equation (49) and equation (35), but if I am missing any subtle details, then perhaps that is indicating that a description would be helpful.
\end{quotation}

We agree that this is a useful summary of our implementation.
We have added a new section (Section 7.3) which describes a pseduo-code implementation
of both the quasi-implicit and nonstandard finite difference scheme.

\begin{quotation}
I think of the sloshing instability as a physical one in which there is a resonance between the waves on a fluid and the dynamics of the container (fluid sloshing in tankers causing capsizes, for example). So I am not sure that it is a good name for a purely numerical instability even if that is how it has been named by others.
\end{quotation}

This point is well taken. We now note that this instability has been called the ``sloshing'' instability,
but in the bulk of the text we refer to it as the ``drunken sailor'' instability.

\begin{quotation}
Another example I can think of where this stabilization might be useful is at an inflow boundary where the inflow condition is used to balance a global contraint (e.g. volume / mass conservation). An example of where this arises is in modeling of extensional systems where isostasy sets the lower boundary to have a constant (average) pressure through time. There are plenty of spurious modes that can arise in the inflow condition but one instability that is often important is similar to the drunken-sailor and can be supressed by small timesteps.
\end{quotation}
This sounds like an interesting application.
Indeed, there are several applications that exhibit similar instabilities, 
and may be good targets for similar stabilization methods 
(including fluid-structure-interactions and elastoplastic deformation).
That being said, we have not analyzed those applications in this context,
and are unprepared to discuss specific commonalities.
Thus we feel this is outside of the scope of this manuscript.


\begin{quotation}
Following on, we sometimes fix those sorts of problems by under-relaxing the boundary deformation over time with very little mathematical justification for doing so. I would be interested to know if this is a poor-man's equivalent to the solution you propose.
\end{quotation}

We would suggest that under-relaxing boundary deformation is indeed a poor-man's equivalent to our NSFD scheme,
and would love to hear more about this practice.
However, we are not aware of discussion of this in the literature or open-souce codebases, and so can't comment further.

\begin{quotation}
   Your method remains formally 1st order. Are higher-order discretisations of the mesh deformation helpful in this case or do they suffer the same / worse problems ?
\end{quotation}

Our analysis of the instability is purely in the time discretization, 
so higher-order spatial discretizations should suffer from the identical instability.
We have added a note to this effect in the manuscript.

\section*{Reviewer 2}
\begin{quotation}
This is useful and timely paper developing new approach for the free surface treatment with finite elements in geodynamic simulations. This is an important topic in geodynamics since essential need for free surface in modeling regional and global processes has been repeatedly demonstrated (e.g. Crameri, F., Tackley, P.J., Meilick, I., Gerya, T., Kaus, B.J.P. (2012) A free plate surface and weak oceanic crust produce single-sided subduction on Earth. Geophys. Res. Lett., doi:10.1029/2011GL050046. And references therein). The revised version reads well and addressed properly the comments of reviewers. I only noticed few minor points that need some attention (listed below).
Taras Gerya, Zurich, 15.08.2016
\end{quotation}

\begin{quotation}
Introduction. The earliest implementation of sticky air approach was published by Matsumoto, T., Tomoda, Y., 1983. Numerical-simulation of the initiation of subduction at the fracture-zone. J. Phys. Earth 31, 183–194.
Introduction. There is one recent paper about true free surface implementation with staggered grid by T Duretz, DA May, P Yamato  (2016) A free surface capturing discretization for the staggered grid finite difference scheme Geophysical Journal International 204 (3), 1518-1530 
\end{quotation}
We have added these pertinent references.

\begin{quotation}
Line 27. “A true free surface has mathematical elegance in that the boundary condition of the domain more closely matches the boundary conditions which one is trying to model, but it typically requires a deformable domain with frequent remeshing to avoid ill-conditioned cells.” One should mention here, however, possibilities of treating more complex realistic cases of common geodynamic simulations, in which oceanic free surface is covered by water whereas continental regions contact with the athmosphere. Density contrast along such free surface will be non-uniform, which would require some special treatments.
\end{quotation}
This is a good point, and we have added a note to this effect.
\begin{quotation}
Equations 1,3,4,5 – minus is missing in front of RHO*g
\end{quotation}
These are now fixed.

\begin{quotation}
The authors mentioned that they did Schmeling et al. benchmark (2008) but do not show it since it is difficult to interpret and quantify. I would still recommend showing one-two figures with the progress of model evolution and surface development for this model. It will simply show that the developed algorithm does not break when treating more complex plate tectonics-like models with large viscosity contrasts and complex evolution and advection of the free surface.
\end{quotation}

We investigated the subduction benchmark of Schmeling et al. 2008, but do not report
the results because the benchmark is not well designed to test our time-stepping scheme.
The chief difficulty is sharp corners in the shape of the initial slab. While the initial condition
is easy to implement, it introduces singularities into the problem. These singularities are
resolved indirectly by limiting the spatial resolution. Moreover the viscosity contrast in this
subduction benchmark is only a factor of 100. Relatively little is gained over the (smaller)
factor of 10 viscosity increase in the Rayleigh-Taylor benchmark.

\end{document}
